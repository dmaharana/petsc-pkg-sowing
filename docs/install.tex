\documentclass[11pt,twoside]{article}
\usepackage{epsf}
\usepackage{doctext/tpage}
\usepackage{/homes/gropp/bin/share/tex/code}
\usepackage{/homes/gropp/bin/share/tex/fileinclude}
\usepackage{/homes/gropp/bin/share/tex/url}

%% This is the MPICH installation guide!
\textwidth=6.0in
\textheight=9.0in
\hoffset=-0.5in
\voffset=-1.0in
\parskip=.1in
\renewcommand{\floatpagefraction}{0.9}
\newcommand{\sowing}{{\tt sowing}}
\newcommand{\Sowing}{{\tt Sowing}}
\def\note#1{\marginpar{#1}}
%
% The following is needed for the acknowledgements contents entry
%
\makeatletter
\def\l@chapter{\hfil}
\makeatother
%
% Directory and definitions for help text
\def\helpdir{/home/MPI/maint/help/src}
\def\question{{\bf Q: }}
\def\answer{{\bf A: }}
\def\ttURL#1{{\tt\URL{#1}}}

\let\DOESupport\DefaultSupport

\begin{document}
\ANLTMTitle{Installation Guide for \code{sowing},\\ Tools for developing and
  distributing programs}{\em William Gropp and Ewing Lusk}{ANL-00/0}{September 1997}

\clearpage
\pagenumbering{roman}
\setcounter{page}{3}
\pagestyle{plain}
{\parskip=0pt
\tableofcontents
\bigskip
\bigskip
This {\em Guide\/} corresponds to Version 0.1 of {\tt sowing}.  It was
processed by \LaTeX\ on \today.
\clearpage
}
\pagenumbering{arabic}
\setcounter{page}{1}

\addcontentsline{toc}{section}{Abstract}
\begin{abstract}
 Tool building and distribution is important.
\end{abstract}

This document describes how to obtain and install
\texttt{sowing}~\cite{sowing}, a collection of portable tools for developing
and 
distributing software.  Details on using each of the tools is provided in
separate users manuals \cite{bfort,doctext,tohtml}.

\section{Quick Start}
\label{sec:quickstart}

Here is a set of steps for setting up and minimally testing \code{sowing}.
Details and instructions for a more thorough tour of \code{sowing}'s features,
including installing, validating, benchmarking, and using the tools, are given
in the following sections.  
\begin{enumerate}
\item If you have \code{gunzip}, get \file{sowing.tar.gz}; otherwise, get 
\file{sowing.tar.Z} by anonymous \code{ftp} from
  \code{ftp.mcs.anl.gov} in the directory \code{pub/sowing}.
\item \code{gunzip -c sowing.tar | tar xovf -} or 
\code{zcat sowing.tar.Z | tar xovf -}
\item \code{cd sowing}
\item \code{configure}  This will attempt to choose an appropriate default
  architecture and ``device'' for you.  If the defaults are not what you want,
  see Section~\ref{sec:configuring}.  If you are \emph{not} going to install
  sowing, give \code{configure} the argument \code{--enable-inplace}.
\item \code{make >& make.log} (in C-shell syntax).  
At this point you have built the sowing programs.  
\item (Optional)  Run the tests
  (See Section~\ref{sec:thorough} for how to do this).  
\item (Optional)  If you wish to install \code{sowing} in a public place so
  that others may use it, use
\begin{verbatim}
  make install
\end{verbatim}
to install \code{sowing}.  Following standard practice with \code{configure},
the directories for the installation are chosen when you use configure with
the \code{--prefix} and \code{--datadir} options.
\end{enumerate}

In the following
sections we go through these steps in more detail, and describe other aspects
of the \code{sowing} distribution you might want to explore.

\section{Obtaining and Unpacking the Distribution}
\label{sec:unpacking}

\code{sowing} can be obtained by anonymous \code{ftp} from the site
\code{ftp.mcs.anl.gov}.  Go to the directory \code{pub/sowing} and \code{get}
the file \code{sowing.tar.gz}.  This file name is a link to the most recent
verstion of \code{sowing}.  Currently it is about 1 Megabyte in size.  The
file is a {\tt gzip}ped tar file, so it may be unpacked with
\begin{verbatim}
    gunzip -c sowing.tar.gz | tar xvf -
\end{verbatim}
If you do not have {\tt gunzip}, but do have {\tt uncompress}, then you must
get {\tt sowing.tar.Z} instead, and use either 
\begin{verbatim}
    zcat sowing.tar.Z | tar xvf -
\end{verbatim}
or 
\begin{verbatim}
    uncompress sowing.tar.Z
    tar xvf sowing.tar
\end{verbatim}
This will create a single directory called \code{sowing}, containing in various
subdirectories the entire distribution, including all of the source code, some
documentation (including this {\em Guide}), \code{man} pages and the \code{sowing}
environment described in Section~\ref{sec:environment}.
In particular, you should see the following files and directories:
\begin{description}
\item[{\tt COPYRIGHT}] Copyright statement.  This code is free but not public
  domain.  It is copyrighted by the University of Chicago.
\item[{\tt Makefile.in}] Template for the \file{Makefile}, which will be
  produced when you   run \code{configure}.
\item[{\tt README}] Basic information and instructions for configuring.
\item[{\tt aclocal.m4}] Used for building \file{configure} from \file{configure.in};
 not needed for most installations.
\item[{\tt bin}] Home for executable files like \code{pstogif} and \code{pstoxbm}.
\item[{\tt configure}] The script that you run to create Makefiles throughout the
  system. 
\item[{\tt configure.in}] Input to \code{autoconf} that produces \code{configure}.
\item[{\tt docs}] Documentaiton, including this {\em Installation Guide\/} and
  the users guides.
\item[{\tt include}] The include libraries.
\item[{\tt lib}] The machine-dependent libraries, after they are built.
\item[{\tt man}] Man pages for \code{MPI}, \code{MPE}, and internal routines.
\item[{\tt ref}] Contains Postscript versions of the \code{man} pages.
\item[{\tt src}] The source code for the portable part of \code{sowing}.
  There are subdirectories for the various tools.
\end{description}

If you have problems, check the \sowing\ home page on the Web at
\gb\ttURL{http://www.mcs.anl.gov/Projects/sowing} .  This page has pointers
to lists of known bugs and patchfiles.  If you don't find what you need here,
send mail to \code{mpi-bugs@mcs.anl.gov}.

\section{Documentation}
\label{sec:documentation}

This distribution of \sowing\ comes with complete \code{man} pages for the
programs.  The command \code{sowingman} in \code{sowing/bin} is a good
interface to the man pages.\footnote{The {\tt sowingman} command is created by
  the configure process described later.}
The \code{sowing/ref} directory contains printable versions
of the man pages for the programs, in compressed PostScript form.
The \code{sowing/docs} directory contains this {\em Installation Guide\/} and
also the users guides.

\section{Configuring {\tt sowing}}
\label{sec:configuring}

The next step is to configure \sowing\ for your particular computing
environment.  \sowing\ can be built for most Unix systems; many of the
programs can also be built for Microsoft Windows.

Configuration of \sowing\ is done with the \code{configure} script contained in
the top-level directory.  This script is automatically generated by the Gnu
\code{autoconf} program from the file \code{configure.in}, but you do not need
to have \code{autoconf} yourself.

The configure script documents itself in the following way.  If you type
\begin{verbatim}
    configure --help
\end{verbatim}
you will get a complete list of arguments and their meanings.  So the
simplest way to document the options for {\tt configure} is to just show
its output here:
\bigskip

\fileinclude{config.txt}

Normally, you should use \code{configure} with as few arguments as you can.

Some sample invocations of \code{configure} are shown below.  In many case,
the detailed invocations below are the defaults, which you would get if you
invoked \code{configure} with no arguments.

To build and test in-place, using strict checking with the Gnu C and C++
compilers: 
\begin{verbatim}
configure --enable-strict --datadir=`pwd`/share --prefix=`pwd`
\end{verbatim}

\section{Compiling \sowing}
\label{sec:compiling}

Once \code{configure} has determined the features of your system, all we have
to do now is 
\begin{verbatim}
    make
\end{verbatim}
This will clean all the directories of previous object files (if any), compile
the source code, build all necessary
libraries, and build the programs.  If anything goes wrong, check
Section~\ref{sec:problems} to see if there is anything said there about your
problem.  If not, follow the directions in Section \ref{sec:bugreports} for
submitting a bug report.  To simplify checking for problems, it is a good idea
to use 
\begin{verbatim}
    make >& make.log &
\end{verbatim}
Specific (non-default) targets can also be made.  See the \code{Makefile} to
see what they are.

After running this {\tt make}, the size of the distribution will be about 3
Megabytes (depending on the particular machine it is being compiled for). 

\section{Thorough Testing}
\label{sec:thorough}
\label{sec-tests}

Each program directory contains a \code{testing} directory contains tests of
the program.
The command
\begin{verbatim}
    make testing
\end{verbatim}
in the \code{sowingt} directory will cause these programs to be
compiled, linked, executed, and their output to be compared with the expected
output.  Linking all these test programs takes up considerable space, so you
might want to do 
\begin{verbatim}
    make clean
\end{verbatim}
afterwards.  

If you have a problem, first check the troubleshooting guides and the lists of
known problems.  If you still need help, send detailed information to
\code{mpi-bugs@mcs.anl.gov}. 

\section{Installing \sowing\ for Others to Use}
\label{sec:installing}
This step is optional.  
Once you have tested all parts of the \Sowing\ distribution, you may install
\sowing\ into 
a publically available directory, and disseminate information for other users,
so that everyone can use the shared installation.
To install the libraries and include files in a
publicly available place, choose a directory, such as \code{/usr/local/sowing},
change to the top-level \code{sowing} directory, and do
\begin{verbatim}
    make install
\end{verbatim}
The \code{man} pages will have been copied with the installation, so you might
want to add \code{/usr/local/mpi/man} to the default system \code{MANPATH}.
The man pages can be conveniently browsed with the \code{mpiman} command,
found in the \code{sowing/bin} directory.

A good way to handle multiple releases of \code{sowing} is to install
them into directories whose names include the version number and then
set a link from \code{sowing} to that directory.  For example, if the
current version is 1.0, the installation commands to install into
\file{/usr/local} are 
\begin{verbatim}
configure --prefix=/usr/local/sowing-1.0
make
make install 
rm /usr/local/sowing
ln -s /usr/local/sowing-1.1.0 /usr/local/sowing
\end{verbatim}


\subsection{Installing documentation}
The SOWING implementation comes with several kinds of documentation.
Installers are encouraged to provide site-specific information, such
as the location of the installation (particularly if it is not in
\file{/usr/local/sowing}).

\subsubsection{Man pages}
A complete set of Unix man pages for the SOWING implementation are in
\file{sowing/man}.  \file{man/man1} contains the sowing programs.
%The command \file{sowing/bin/sowingman} is a script that runs \code{xman}
%on these man pages.

\subsubsection{Web versions of man pages}
Web (HTML) versions are available from
\URL{ftp://ftp.mcs.anl.gov/pub/sowing/sowingwww.tar.Z}.  They are available
at \URL{http://www.mcs.anl.gov/sowing/www}.
A sample Web page is shown below, and is also available in
\file{sowing/docs/sowingsite.html} 

\fileinclude{/home/gropp/sowing-proj/sowing/docs/sowingsite.html}

\section{Problems}
\label{sec:problems}

This section describes some commonly encountered problems and
their solutions.  It also describes machine-dependent considerations.
You should also check the Users Guide, where problems related to compiling,
linking, and running MPI programs (as opposed to building the SOWING
implementation) are  discussed.

\subsection{Submitting bug reports}
\label{sec:bugreports}
Any problem that you can't solve by checking this section should be sent to
\code{mpi-bugs@mcs.anl.gov}.  
Please include:
\begin{itemize}
\item     The version of SOWING (e.g., 1.0)

\item     The output of 
\begin{verbatim}
        uname -a
\end{verbatim}
     for your system.  If you are on an SGI system, also
\begin{verbatim}
        hinv
\end{verbatim}

\item     If the problem is with configure  run with the \code{--enable-echo} argument 
     (e.g., \code{configure --enable-echo} )

\end{itemize}
If you have more than one problem, please
send them in separate messages; this simplifies our handling of problem
reports. 
 
The rest of this  section contains some information on trouble-shooting
\sowing. 
Some of 
these describe problems that are peculiar to some environments and give
suggested work-arounds.
Each section is organized in question and answer format, with questions that
relate to more than one environment (workstation, operating system, etc.)
first, followed by questions that are specific to a particular environment.
Problems with workstation clusters are collected together as well.

\subsection{Problems configuring}

\subsubsection{General}
\begin{enumerate}
% Primative shell (no procedures)
\item \input{\helpdir/all/m1.txt}

\item 
\question 
The configure reports the compiler as being broken, but there is
no problem with the compiler (it runs the test that supposedly failed without
trouble).

\answer 
You may be using the Bash shell (\file{/bin/bash}) as a replacement for
the Bourne shell (\file{/bin/sh}).  We have reports that, at least under
LINUX, Bash does not properly handle return codes in expressions.  One fix is
to use a different shell, such as \file{/bin/ash}, on those systems.

This won't work on some LINUX systems ({\em every} shell is broken).  We have
reports that the following will work:
\begin{enumerate}
\item In \file{configure}, change \code{trap 'rm -f confdefs*' 0} to 
\code{trap 'rm -f confdefs*' 1}
\item After configure finishes, remove the file \file{confdefs.h} manually.
\end{enumerate}

\item 
\question 
configure reports errors of the form
\begin{verbatim} 
checking gnumake... 1: Bad file number
\end{verbatim}

\answer 
Some versions of the \code{bash} shell do not handle output redirection
correctly.  Either upgrade your version of \code{bash} or run configure under
another shell (such as \code{/bin/sh}).  Make sure that the version of
\code{sh} that you use is not an alias for \code{bash}.  \code{configure} 
trys to detect this situation and will normally issue an error message.

\end{enumerate}

\subsection{Problems building {\tt sowing}}
\subsubsection{General}
\begin{enumerate}
\item 
\question 
When running make on \code{sowing}, I get this error:
\begin{verbatim}
ar: write error: No such file or directory
*** Error code 1
\end{verbatim}
I've looked, and all the files are accessible and have the proper permissions.

\answer 
Check the amount of space in \file{/tmp}.  This error is sometimes
generated when there is insufficient space in \file{/tmp} to copy the archive
(this is a step that \code{ar} takes when updating a library).  The command
\code{df /tmp} will show you how much space is available.  Try to insure that
at least twice the size of the library is available.

\item \question
When running make on \code{sowing}, I get errors when executing \code{ranlib}.

\answer
Many systems implement \code{ranlib} with the \code{ar} command, and they use
the \file{/tmp} directory by default because it ``seems'' obvious that using
\file{/tmp} would be faster (\file{/tmp} is often a local disk).
Unfortunately, a large number of systems have ridiculously small \file{/tmp}
partitions, and making any use of \file{/tmp} is very risky.  In some
cases, the \code{ar} commands used by SOWING will succeed because they 
use the \code{l} option---this forces \code{ar} to use the local directory 
instead of \file{/tmp}.  The \code{ranlib} command, on the other hand, may use
\file{/tmp} and cannot be fixed.

In some cases, you will find that the \code{ranlib} command is unnecessary. 
In these cases, you can reconfigure with \code{-noranlib}.  If you must use
\code{ranlib}, either reduce the space used in \file{/tmp} or increase the
size of the \file{/tmp} partition (your system administrator will need to do
this).  There should be at least 20--30 MBytes free in \file{/tmp}.

\item
\question
When doing the link test, the link fails and does not seem to find any of the
sowing system routines:
\noindent
\begin{verbatim}
  /homes/them/burgess/sowing/lib/IRIX32/ch_p4/mpicc \
                                         -o overtake overtake.o test.o 
  ld: WARNING 126: The archive \
     /homes/them/burgess/sowing/lib/IRIX32/ch_p4/libmpi.a \
                                   defines no global symbols. Ignoring.
  ld: WARNING 84: /usr/lib/libsun.a is not used for resolving any symbol.
ld: ERROR 33: Unresolved data symbol "MPI_COMM_WORLD" -- \
                                          1st referenced by overtake.o.
ld: ERROR 33: Unresolved text symbol "MPI_Send" -- \
                                          1st referenced by overtake.o.
...
\end{verbatim}

\answer
Check that the \code{ar} and \code{ranlib} programs are compatible.  One site
installed the Gnu \code{ranlib} in such a way that it would be used with the
vendors \code{ar} program, with which it was incompatible.  Use the
\code{-noranlib} option to \code{configure} if this is the case.

\end{enumerate}

\subsubsection{SGI}
\begin{enumerate}
\item 
\question 
The build on an SGI Power Challenge fails with 
\begin{verbatim}
Signal: SIGSEGV in Back End Driver phase.
> ### Error:
> ### Signal SIGSEGV in phase Back End Driver -- processing aborted
> f77 ERROR:  /usr/lib64/cmplrs/be died due to signal 4
> f77 ERROR:  core dumped
> *** Error code 2 (bu21)
> *** Error code 1 (bu21)
> *** Error code 1 (bu21)
\end{verbatim}

\answer Our information is that setting the environment variable \code{SGI_CC} to
\code{-ansi} will fix this problem.

\end{enumerate}

\subsubsection{DEC ULTRIX}
\begin{enumerate}
\item
\question
When trying to build, the \code{make} aborts early during the cleaning phase:
\begin{verbatim}
amon:SOWING/sowing>make clean
        /bin/rm -f *.o *~ nupshot
*** Error code 1
\end{verbatim}

\answer
This is a bug in the shell support on some DEC ULTRIX systems.  You may be
able to work around this with

\begin{verbatim}
setenv PROG_ENV SYSTEM_FIVE
\end{verbatim}
Configuring with \code{-make=s5make} may also work.

\end{enumerate}


\section*{Acknowledgments}
\addcontentsline{toc}{section}{Acknowledgments}

The work described in this report has benefited from conversations with and
use by a large number of people.
Particular thanks goes to Barry Smith and Lois McInnes for valuable help
in the implementation and development of sowing.

\bibliographystyle{plain}
\bibliography{paper}

\end{document}




