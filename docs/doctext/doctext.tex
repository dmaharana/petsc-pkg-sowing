%\tracingall
\documentclass[twoside]{linfoem}
\c\usepackage[dvipdfm]{hyperref}
\usepackage{epsf,refman2,tpage}
\usepackage{/homes/gropp/bin/share/tex/fileinclude}
\c\documentstyle[epsf,refman2,twoside,tpage,addtensf,latexinfo]{article} 
\c If you have latex2e, then you don't have a complete simulation of LaTeX.
\c The following detects that and tries to fix it.
\c Unfortunately, this needs to be done BEFORE loading latexinfo.  And since
\c rawfonts isn't part of LaTeX (only latex2e), we can't use it in the
\c document style command.
\c \ifx\tensf\relax\usepackage[only,tensf]{rawfonts}\fi
\c \documentstyle[epsf,../refman2,twoside,../funclist,tpage,../tools,../../tex/latexinfo]{article} 
\c -*-latexinfo-*-
\c \anltmtrue
\anltmfalse
\c\pagestyle{empty}
\c
\c We really need a ``usage'' macro for the routines and macros
\def\usage#1{}
\c
\c the definition of \file should be changed so that it adds an
\c \penalty 1000\hskip 0pt +1fil \penalty -10000
\c before the actual text (allowing a line that the file name would stick
\c out of to instead be raggedright, a not perfect but better approach).
\c
\c
\c See /home/beumer/GROPP/TMS/doctext.tex
\c
\let\DOESupport\DefaultSupport

\c\def\tie{\nobreak\ \nobreak}
\def\bw{{\tt\char`\\}}
% Give the URL for a document (WWW version makes URL active)
\def\URL#1{#1}

\textheight=9in
\textwidth=6.5in
\oddsidemargin=0in
\evensidemargin=0in
\topmargin=-0.5in
\hsize=\textwidth

\begin{document}
\c latex2e's compatibility mode wants this
\c \makeatletter
\c  \global\let \@nodocument \relax
\c \makeatother
\c 
\newindex{fn}

\begin{ifinfo}
\c %**start of header
\title{Users Manual for doctext:\\
Producing Documentation from Source Code}
\author{William Gropp\\
Mathematics and Computer Science Division\\
Argonne National Laboratory}

\date{\today}

\c %**end of header

\maketitle
\end{ifinfo}

\begin{iftex}
\ANLTMTitle{Users Manual for doctext:\\
Producing Documentation from Source Code}{\em William Gropp\\
Mathematics and Computer Science Division}{206-Rev 3}{February 1999}
\c {March 1995}
\c {Rev 1 April 1997}

\end{iftex}

\clearpage

\c \vskip 0pt plus 1filll
\c Copyright \copyright{} 1995 Argonne National Laboratory
\c
\c \clearpage
\c \pagenumbering{roman}
\c \setcounter{page}{3}
\c \pagestyle{plain}
\c \tableofcontents
\c \clearpage

\pagenumbering{arabic}
\pagestyle{plain}
\thispagestyle{plain}

\setfilename{../doctext.info}

\node Top,        Introduction, (dir), (dir)
\c    nodename,   next,          previous, up

\c ---------------------------------------------------------------------------
\begin{iftex}

\pagenumbering{arabic}
\setcounter{page}{1}
\c \thispagestyle{empty}

\begin{center}
{\large\bf Users Manual for doctext:\\
Producing Documentation from Source Code\\}
\vspace{.2in}
by \\
\vspace{.2in}
{\it
William Gropp\\
}
\end{center}

\vspace{.2in}

\addcontentsline{toc}{section}{Abstract}
\begin{abstract}
\noindent 
One of the major problems that software library writers face, particularly in
a research environment, is the generation of documentation. 
Producing good,
professional-quality documentation is tedious and time consuming.  
Often, no documentation is
produced.   For many users, however, much of the need for
documentation may be satisfied by a brief description of the purpose and use
of the routines and their arguments.  Even for more complete,
hand-generated documentation, 
this information provides a convenient starting point.   We
describe here a tool that may be used to generate documentation about programs
written in the C language.  It uses a structured comment convention that
preserves the original C source code and does not require any additional
files.  The markup language is designed to be an almost invisible 
structured comment in the C source code, retaining readability in the
original source.
Documentation in a form suitable for the Unix man program (nroff), LaTeX, and
the World Wide Web can be produced.  
The output format is controlled by easily modified tables, permitting
customization of the output.
Support for other languages is also provided, though with fewer features.
\end{abstract}
\end{iftex}


\clearpage
\c ---------------------------------------------------------------------------



\c ---------------------------------------------------------------------------

\node Introduction,doctext,Top,Top
\section{Introduction}
\label{chap:intro}
\c Generating good documentation is a task that requires skill and hard
\c work.  
The tools described in this report are intended to help you create
simple man-page-style documentation quickly and easily. 
Specifically, the program \code{doctext} takes C programs
and generates nroff (Unix man page format), LaTeX \cite{Lamport86}, or HTML
files.  All of the 
information is  
embedded in structured comments, allowing the documentation to be maintained
along with the code.
The design of the special markup language commands emphasizes
readability of the original source code (the structured comment).  This
approach differs from approaches such as \code{web} \cite{web}, where
special processing must be applied to produce a readable version of the
source code. Such approaches can do much more than \code{doctext} but
are really aimed at documenting the source code of a program rather than
the use of the program.  Further, because those approaches are so 
intrusive, users tend to avoid them.  The approach in \code{doctext} provides
an effective comprimise between usability and functionality.

The \code{doctext} approach has been independently adopted by Java; the
\code{javadoc} 
program provides HTML documentation from Java source.  \code{doctext} predates
Java and provides more formats (LaTeX and nroff as well as HTML) and
formatting commands.

\node The doctext Program,,,Top
\section{The doctext Program}
The \code{doctext} program reads C programs and
generates documentation.  Command-line options provide control over the
output.  Fine tuning of the output can be accomplished by changing files that
define the output form for each of the commands that \code{doctext} recognizes.

\node Getting Started,,,The doctext Program
\subsection{Getting Started}
Using \code{doctext} is easy.  For example, the command
\begin{example}
doctext foo.c
\end{example}
will generate Unix man pages for all of the commented routines in the file
\file{foo.c}.
But before this will do you any good, you will need to add some structured
comments to your file.

\node Structure Comments,,,The doctext Program
\subsection{Structured Comments}
The \code{doctext} program searches for C comments of the form \code{/*}{\em c} \code{...}
{\em c}\code{*/}, where {\em c} is a single character indicating the type of
documentation.  The available types are \code{@} for
routines, \code{M} for macros, \code{E} for enum definitions, 
\code{S} for struct definitions, and \code{D} for
other miscellaneous documentation (such as introductions or descriptions of
programs rather than routines).
In all cases, the structured comment has this form:
\begin{example}
/*@ 
    name - short description

    heading 1:

    heading 2:

    ...
@*/
\end{example}
The structured comment for a routine must immediately precede the declaration
of the routine (either K\&R or ANSI-style prototypes).
Figure \ref{fig:example} shows the structured comment and the routine being
documented.

\begin{figure}
\begin{example}
/*@
   Swap - Swaps two pointers

   Parameters:
.  ptr1,ptr2 - Pointers to swap
@*/
void Swap( ptr1, ptr2 )
void **ptr1, **ptr2;
\{
void *tmp = *ptr1;
*ptr1 = *ptr2;
*ptr2 = tmp;
\}
\end{example}
\caption{C code for a Swap routine with \protect\code{doctext}-style
structured comment}\label{fig:example} 
\end{figure}

The man (nroff-style) output of this is shown in Figure \ref{fig:swap-man},
and the LaTeX output is shown in Figure \ref{fig:swap-latex}.

\begin{figure}
\fileinclude{swap.man}
\c \centerline{\epsfysize=3in\epsfbox{swap2.eps}}
\caption{Unix-style man page for the Swap routine}\label{fig:swap-man}
\end{figure}

\begin{figure}
\begin{minipage}{\hsize}
\input Swap.tex
\end{minipage}
\caption{LaTeX page for the Swap routine}\label{fig:swap-latex}
\end{figure}

The body of the structured comment follows simple rules.  Any line that ends
in a colon (\code{:}) generates a section title with the line as the title.
In the example of \code{Swap}, the line \code{Parameters:} generates a section
in the man page with title {\bf Parameters}. To end a line in a colon without
generating a section heading, precede it with a backslash:
\begin{example}
   This is not a new section\bw:
\end{example}

The first column within a structured comment has a special meaning.  A period
followed by a space indicates that the line 
begins the description of an argument (ended by another argument or a blank
line).  This line has a special format.  The first space-separated token is
taken as the argument.  The next character should be a dash (\code{-}).  After
the dash comes the text.  The \code{Swap} example shows this for the arguments
\code{ptr1,ptr2}.
Other commands are described below in Section \ref{sec-formatting}.

\node C Routines,,,The doctext Program
\subsection{C Routines}
C routines are indicated by the structured comment \code{/*@ ... @*/}.
The \code{doctext} program provides a synopsis automatically by reading the declarations
of the routine.  
Arguments are specified as described in \xref{Describing Arguments}.  

\node C Macros,,,The doctext Program
\subsection{C Macros}
C macros are indicated by the structured comment \code{/*M ... M*/}.
Unlike the case of C routines, macro definitions do not provide any
information on the types of the arguments.
Thus, the body of a structured comment
for a C macro 
should include a {\em synopsis} section, containing a declaration of the
macro as if it were a C routine.  For example, if the \code{Swap} example were
implemented as a macro, the structured comment for it would look like
\begin{example}
/*M
   Swap - Swaps two pointers

   Parameters:
.  ptr1,ptr2 - Pointers to swap

   Synopsis:
   void Swap( ptr1, ptr2 )
   void **ptr1, **ptr2;   
M*/
\end{example}
It is important that the word {\em Synopsis} be used; \code{doctext} and
related programs (\code{bfort} and \code{doc2lt}) use this name to find the
C-like declaration for the macro. 

\subsection{Enums and Structs}
A C enum may be documented using the \code{E} command:
\begin{example}
/*E
  Read_t - Enum for read modes
E*/
typedef enum { Read, ReadBack } Read_t;
\end{example}

A C struct definition uses the \code{S} command:
\begin{example}
/*S
  File_t - Structure defining a file type
S*/
typedef struct { 
    Read_t mode;
    int    fd;
    int (*readfn)( int, void *, int );
    int (*writefn)( int, void *, int );
    } File_t;
\end{example}

\node Miscellaneous Documentation,,,The doctext Program
\subsection{Miscellaneous Documentation}
In addition to routines, a library will often have a few additional manual
pages, for example, an overview of the members of the library or instructions
on installing or debugging the library.  In order to allow the same tools to
be used 
for all of the documentation, a comment of the form \code{/*D ... D*/} may
appear anywhere and will generate a manual page.  

\node Indicating Special Limitations,,,The doctext Program
\subsection{Indicating Special Limitations}
Two modifiers to the structured comments indicate special 
behavior of the function.  The modifiers must come after the character that
indicates a routine, macro, or documentation.
The modifier \code{C} indicates that this routine is available only in C (and
not from Fortran).  For example, the \code{Swap} program cannot be used in
Fortran, so its structured comment should be
\begin{example}
/*@C
  ....
@*/
\end{example}

The modifier \code{X} indicates that the routine requires the
X11 Window System.  This is intended primarily for the program \code{bfort}
\cite{bfort}, which is used to generate Fortran interfaces for systems that do
not have X11. 

The modifiers \code{C} and \code{X} may be used together and may be specified
in either order (i.e., \code{CX} or \code{XC}).

\node Indicateing Include Files,,,The doctext Program
\subsection{Indicating Include Files}
It is often very important to indicate what include files need to be used with
a particular routine.  This may be accomplished with a special structured
comment of the form \code{/*I}{\em include-file-name}\code{I*/}.
For example, to indicate that the routine requires that \file{<sys/time.h>}
has been included, use
\begin{example}
#include <sys/time.h>          /*I <sys/time.h> I*/
\end{example}
in the C file.
A user-include can be specified as
\begin{example}
#include "system/nreg.h"      /*I "system/nreg.h" I*/
\end{example}
This approach of putting the structured include comment on the same line as
the include of the file ensures that if the source file is changed by removing
the include, the documentation will reflect that change.
Includes are added to the synopsis of all routines in the file that
contains the include comment.

\node Special Formatting,,,Top
\section{Special Formatting}\label{sec-formatting}
The structured comment format for \code{doctext} was designed to have a small
impact on the appearance of the C source code.  As such, it provides minimal
formatting for the generated manual pages.  
Three important cases
are provided: describing
arguments, verbatim output, and common blocks of text.  
In addition, there are some additional formatting controls for things like
emphasis (italics text) and fixed-width fonts.
This section describes all of these in more detail.

\node Describing Arguments, , ,Special Formatting
\subsection{Describing Arguments}
Arguments to routines and command-line options for programs are described by
using a plus (\code{+}), period (\code{.}), and minus (\code{-})
in the first column, followed by one or more spaces.
The name of the argument or command line variable is next.
Several arguments may be placed together, separated with spaces and/or commas.
To include a hyphen (\code{-}) in the argument, 
prefix it with a backslash (\bw).
Follow this with one or more spaces, then a dash (\code{-}), optional spaces,
and finally the 
text of the description.   The description may use multiple lines.
The \code{+} should be used to start an argument list and \code{-} should be
used to end the list.  The period (\code{.}) should be used for all other
arguments in the list.
For example,
\begin{example}
+ a - Pointer to
      the data item
. i, j - Sizes of data item
. str - Another argument value
- \bw-arg value - Sample command-line argument
\end{example}
The \code{+} and \code{-} are optional but recommended.  They permit better
formatting of lists of arguments.  

\node Common Blocks of Text, , ,Special Formatting
\subsection{Common Blocks of Text}
In some cases, it is useful to repeat a body of text in many man pages.  For
example, there may be a common argument that has a lengthy description.  A
common block of text is defined with 
\begin{example}
/*N name
   text
N*/
\end{example}
The {\em name} may be any blank-delimited string.  

To insert this block of text into a man page, use the \code{.N} format command
\begin{example}
/*@
    ...

.N name

    ...
@*/
\end{example}
The definition must precede all uses.  Once defined, the definition is
remembered; if multiple files are processed and the first file contains a
definition of a named block, all subsequently processed files may refer to
that named block.
This feature allows different replacement texts for different situations.  For
example, the replacement text for a man page should contain the entire
text, whereas the replacement text for a manual (LaTeX) may reference a
common section.  

For example, the MPICH implementation of MPI uses this to define all of the 
error conditions.  A file \file{errnotes} contains lines like
\begin{example}
/*N Errors
Errors:

 ... text about MPI errors ...

N*/
/*N MPI_SUCCESS
. MPI_SUCCESS - No error; MPI routine completed successfully.
N*/
/*N MPI_ERR_BUFFER
. MPI_ERR_BUFFER - Invalid buffer pointer.  Usually a null buffer where
  one is not valid.
N*/
\end{example}
Each MPI man page ends with references to the name blocks for the errors that
the particular MPI routine may return.  \code{MPI_Send} has the
following commands at the end of its structured comment:
\begin{example}
.N Errors
.N MPI_SUCCESS
.N MPI_ERR_COMM
.N MPI_ERR_COUNT
.N MPI_ERR_TYPE
.N MPI_ERR_TAG
.N MPI_ERR_RANK
\end{example}


\node Line Breaks, , ,Special Formatting
\subsection{Line Breaks}
A new line may be begun with the \code{.n } command at the beginning of a
line.  The rest of the line is read.  For example, to provide a list of items,
use
\begin{example}
.n This is the first item
.n This is the second item
.n This is the third item
.n
\end{example}
Note that a final line containing only \code{.n} is used to ensure that
the next line of text begins on a new line rather than at the end of the
third line.

\node Verbatim Blocks of Text, , ,Special Formatting
\subsection{Verbatim Blocks of Text}
To display a block of text in a fixed-width font, use the \code{.vb}
command to begin the block and \code{.ve} command to end the block.  
These commands, like the other dot commands, must begin in the first
column.  For example, to display two lines of shell commands, use this
text in the structured comment:
\begin{example}
.vb
  cd .
  echo "I'm here"
.ve
\end{example}

\node Emphasis, , ,Special Formatting
\subsection{Emphasis}
To emphasize some text, include it between back quotes: \code{`this text
is emphasized`}.  Emphasized text may appear in italics ({\em this text
is emphasized}) or in bold face ({\bf this text is emphasized}).
(Whether italics or bold face is used depends on the implementation and the
output format).
This effect requires the use of the \code{-quotefmt} command-line option; if
the option is not used, back quotes are intrepreted as a simple character.
To get a single back-quote when \code{-quotefmt} is used, double the quote:
\code{``}.

\node Code and Filename Font, , ,Special Formatting
\subsection{Code and Filename Font}
To display some text in a font appropriate for code and filenames,
include it between single forward quotes: \code{'this text is code'}.
This will appear in a form such as {\tt this text is code}.
This effect requires the use of the \code{-quotefmt} command-line option.
To get a single quote when \code{-quotefmt} is used, double the quote:
\code{''}.

\node Pictures, , ,Special Formatting
\subsection{Pictures}
This section described a proposed feature; comments are welcome. 

In the HTML and LaTeX formats, pictures (Postscript files) can be
included by using the \code{.p} command:
\begin{example}
 .p filename
\end{example}
The file must be a Postscript file.
A caption for the picture can be included with the \code{.cb} and
\code{.ce} command.  This caption is not displayed in the Unix man-page
document but does appear in the LaTeX and HTML versions.  For example,
\begin{example}
 .p sample-out.ps
 .cb
  This image shows the output of the 'foo' command
  Note particularly ...
 .ce
\end{example}
(not yet implemented)

\node Verbatim, , ,Special Formatting
\subsection{Verbatim}
A dollar (\code{$}) in the first column indicates a verbatim line. In most
situations, this can be used for forcing a line break at the end of the line.
An example is
\begin{example}
$ This is a test
$ 
$    Here is a sample command line
$
$ And the start of the description about
  it.
\end{example}
This version of verbatim is deprecated and supported for backward
compatibility. 

\node Keywords, , ,Special Formatting
\subsection{Keywords}
Keywords may be specified with the command \code{.k}{\em list of keywords}.
For example, to indicate that keywords {\em test, example, help} belong to the
current routine or item, use
\begin{example}
.k test, example, help
\end{example}
To improve the readability of the source, you may use \code{.keywords} instead
of just \code{.k}.

The command line option \code{-keyword filename} causes the keywords to be
appended to the specified file; this can be used to produce a keyword list.

\node Other Formatting Options, , ,Special Formatting
\subsection{Other Formatting Options}
The program \code{doctext} is designed so that it is easy to add additional
formatting commands of the form \code{.}{\em command}.  
In fact, you can define your own formatting commands.  This is done in two
steps.  First, you add formatting commands using a command definition file
(See \xref{Customizing the Output Format}) and include those definitions with
the \code{-defn   filename} command-line argument.  You can then invoke your
new commands by using \code{.f<command name>} in the source file.  For
example, if the file \file{indent.def} contains
\begin{example}
     in \bw{}begin\{quotation\}%n
     out \bw{}end\{quotation\}%n
\end{example}
you can use 
\begin{example}
     .fin 
       This text will be within a LaTeX quotation environment
     .fout
\end{example}
in the source file.

Other commands beyond those described in this document may be
added in the future, so you should not rely on any particular behavior of
\code{.}{\em character}.  For example, \code{.seealso} has been added to
provide a common reference for related information.
If you need some particular formatting
option that you cannot achieve with the \code{.f} approach, please send the
suggestion to \code{gropp@mcs.anl.gov}. 

\node Command Line Arguments,,,Top
\section{Command Line Arguments}
To use \code{doctext}, you need only to give it the name of the files
to process:
\begin{example}
doctext *.[ch]
\end{example}
Command-line options to \code{doctext} allow you to change the details of
how 
\code{doctext} generates output.

A complete list of the command line options follows.  Some of these will
be used often (e.g., \code{-ext} and \code{-mpath}); others are
needed only in special cases (e.g., \code{-outfile}).

The choice of output format is selected with these switches:
\begin{description}
\item[-latex]
Generate LaTeX output rather than man (nroff) format files.
\item[-html]
Generate HTML (WWW) output rather than man (nroff) format files.
\end{description}
If neither of these is used, the output is in man (nroff) format.

The following options control some aspects of the appearance and content of the
generated page, as well as the name of the output file.
\begin{description}
\item[-mpath path]
Set the path where the man pages will be written.
\item[-ext n]
Set the extension (1-9,l, etc.).  The default value is 3.  This is the man page
chapter and is used as the extension on the filename.  For example,
\code{doctext} creates the file \file{Swap.3} for the \code{Swap} routine.
\item[-heading name]
Name for the heading (middle of top line).  The default value is \code{PETSc}.
This is used only for Unix man-page (nroff) format.
\item[-nolocation]
Suppress the generation of a line that indicates the source file that
contained the structured comment.
\item[-quotefmt]
Enable the use of single quotes and back quotes to use code and emphasis
fonts.
\item[-dosnl]
End every line with return-newline instead of just newline.  This makes that
file easier to read on Windows systems, and is quite helpful for HTML output.
\item[-defn filename]Load the output definitions in file \file{filename}, in
  addition to the defaults.  
\item[-I filename]
\code{filename} contains the public includes needed by these routines; it will
automatically be added to the synopsis of the routines.  The file should
contain exactly the text to be added to the synopsis.  For example, to specify
that the include \code{<stdio.h>} be included, use a file with only the text
\code{#include <stdio.h>} in it.
\end{description}

The following options control the generation of additional data about
HTML-format 
output that may be used by other programs to automatically generate links to
the files created by \code{doctext}.
\begin{description}
\item[-index filename]
Generate an index file appropriate for \code{tohtml} \cite{tohtml} (for
generating WWW files). 
This {\em appends} to any existing file; make sure that you don't add
duplicate entries.
\item[-indexdir dirname]
Set the root directory for the index file entries.  This allows you to 
specify an absolute URL for the generated files to be used in the generated
index,  for example, 
\begin{example}
-index foo.cit -indexdir "http:/www.mcs.anl.gov/home/gropp/man"
\end{example}
\item[-mapref filename]Use file \file{filename} as a source of names that
  should be automatically mapped to URLs.  One source of a properly formatted
  mapref file is the output from \code{doctext} in the file specified with the
  \code{-index filename} command.
\c \item[-jumpfile filename]
\c Generate a jump file appropriate for emacs-19 and the command 
\c lookup-petsc-routine-from-text.  
\end{description}

The following options control the output filenames and the list of files to process.
\begin{description}
\item[-outfile filename]
Put the man pages in the indicated file.  Normally, a separate file is
created for each routine.  This may be appropriate for HTML output.
The output filename is normally made up of the name of the routine, followed
by the extension (\code{-ext n}) for nroff, \code{.tex} for LaTeX, or
\code{.html} for HTML files.
\item[-keyword filename]
Append the keywords (from \code{.k} commands) to the given file.  If the
filename is \file{-}, use standard output.
\end{description}


The following options control the input format of the file, allowing
\code{doctext} to be used with documentation in source files for Fortran or
shell scripts.
\begin{description}
\item[-skipprefix name]Set the characters to be skipped at the beginning of
  each line.  The default is the empty string.  Use 
\begin{example}
    -skipprefix C
\end{example}
for Fortran 77 programs.  You can use \code{-skipprefix \#} for shell scripts
and \code{-skipprefix \bw!} for Fortran 90 programs.

For example, with \code{-skipprefix C}, a Fortran program can use the D (not
@) mode:
\begin{example}
C/*D
C fortranfoo - An example of using Fortran with doctext
C
C Input Parameters:
C
C+ foo - bar
C. bar - stool
C- stool - pidgeon
CD*/
      function fortranfoo( foo, bar stool )
      integer foo, bar stool
      end
\end{example}

This option is new and has not been extensively tested.  

\item[-ignore name]Skip over \code{name} when processing a function
  definition.  This is particularly useful when a special (non-standard)
  keyword is needed to build a dynamic link library (DLL) (e.g., Microsoft
  C/C++) or 
  to simplify the construction of export lists (IBM's AIX).  For example, say
  that \code{EXPORT} is used to indicate a routine that will be exported as
  part of a DLL.  The source code then looks like
\begin{example}
/*@ myroutine - a routine visible as part of a DLL
 @*/
EXPORT void myroutine( int a )
\end{example}
Using the command line option \code{-ignore EXPORT} will cause \code{doctext}
to skip the \code{EXPORT} keyword; the synopsis will contain only 
\begin{example}
    void myroutine( int a )
\end{example}

\end{description}

After the command-line arguments come the names of the files from which
documents are to be extracted.

\node Man Pages for Hypertext Documents,,,Top
\section{Man Pages for Hypertext Documents}
One of the most difficult tasks in creating extensive hypertext is generating
the initial documents and providing an easy way to link to them.  
For the the hypertext to be useful, there must be an easy
way to create links to the generated documents.  This section describes how to
do this.  The information produced by \code{doctext} may be used by
\code{tohtml} \cite{tohtml} to generate documents that have extensive
cross-links to 
hypertext versions of their man pages.
Special formatting options for HTML pages are also discussed.

\node Generating Collections of Linked Web Pages,,,Man Pages for Hypertext Documents
\subsection{Generating Collections of Linked Web Pages}
To generate HTML man pages for a collection of source files in
\file{/home/me/foo}, 
do the following:
\begin{verbatim}
cd
mkdir www
mkdir www/man3
cd foo
doctext -html -index ../foo.cit \
        -indexdir http://www.mcs.anl.gov/me/foo/www/man3 \
        -mpath ../www/man3 *.[ch]
cd ..
\end{verbatim}
This puts the HTML files into the directory \file{www/man3} and the index (in
the correct format for the \code{-mapman} option of \code{tohtml}) into the
file 
\file{foo.cit}.   
The \code{-indexdir} option is used to specify the ultimate location for the
files (in this case, the directory \code{me/foo/www/man3} at the Web site
\code{www.mcs.anl.gov}). 
Once you are
sure that the files are correct, you can move them into the Web area with
\begin{example}
cp -r www /mcs/www/home/me
\end{example}
(assuming that \file{/mcs/www} corresponds to \code{http://www.mcs.anl.gov} in
the \code{-indexdir} argument).

To generate an HTML listing of the routines, you can execute the following
script, with, of course, the appropriate changes to the text:
\begin{verbatim}
#! /bin/sh
echo '<TITLE>Web pages for My Routines</TITLE>' >> www/index.html
echo '<H1>Web pages for My Routines</H1>' >> www/index.html
echo '<H2>My Routines</H2>'  >> www/index.html
echo '<MENU>' >> www/index.html
ls -1 www/man3 | \
    sed -e 's%^\(.*\).html$%<LI><A HREF="man3/\1.html">\1</A>%g' \
        >> www/index.html
echo '</MENU>' >> www/index.html
\end{verbatim}
This example may be found in the file \file{mkhtml.sam} in the source
directory for \code{doctext}.

If you have only a few routines to document, you can dispense with the second
directory level above (the \file{man3}).  However, you might find it valuable
to follow (at least loosely) the Unix man page format, with commands and
installation instructions in \file{man1} and routines spread across
\file{man2} through \file{man8}.

\node Alternate Formatting for HTML,,,Man Pages for Hypertext Documents
\subsection{Alternate Formatting for HTML}
\label{sec:alternatehtml}
The appearance of arguments can be improved in HTML by using an HTML table.
The command line argument \code{-defn htmlargtbl.def} will cause a table to be
used in HTML mode for argument lists that begin with \code{+} and end with
\code{-}.  

You can also completely control the type of HTML that is generated.
\code{doctext} uses definition files that define what HTML is produced for
each \code{doctext} command.  These files, for HTML, are \file{html.def} in
the directories specified by \code{TEXTFILT_PATH} (for basic operations) and
\code{DOCTEXT_PATH} (for doctext-specific operations).  These are defined when
\code{doctext} is built; the defaults are \file{/usr/local/share} and
\file{/usr/local/share/doctext}.  You can override any
of these definitions by creating your own file and specifying it with the
\code{-defn filename} argument.  In addition, you can prepend or postpend your
own commands to the predefined commands.  For example, to add an index link to
the bottom of every page, put the following definition in a file:
\begin{example}
+eop    %n<BR><A HREF="http://www.mcs.anl.gov/myindex">Index</A>
\end{example}
This says to prepend (the \code{+} is before the end-of-page command
\code{eop}) the HTML code given.  The \code{%n} generates a newline.
Placeing a \code{+} after the commands postpends the definition.

\node Making a Reference Manual,,,Top
\section{Making a Reference Manual}
With the \code{-latex} option of \code{doctext}, it is easy to create a
reference manual containing the manual pages produced by \code{doctext}.
The style file \code{refman.sty} (in \file{share/refman.sty}) provides the
necessary LaTeX definitions.  Below is an example skeleton for a reference
manual: 

\begin{verbatim}
\documentstyle[refman]{article}

... page size commands ... 

\begin{document}

... title page commands ...

\pagenumbering{roman}
\tableofcontents
\clearpage

\pagenumbering{arabic}
\pagestyle{headings}

\section{Introduction}
... regular text ...

\section{My commands}
\input ref/ref1

... additional sections ...
\end{document}
\end{verbatim}

To generate the LaTeX files for the individual man pages using
\code{doctext}, you can use the following commands:
\begin{example}
doctext -latex -mpath ref/ref1 *.c
\end{example}
followed by
\begin{example}
cd ref
ls ref1 | sort | sed 's/^/\bw input /g' > ref1.tex
\end{example}

If named replacement blocks of text (\code{.N name}) are used, it may be
appropriate to use a different definition for the manual from that used
for the man page.  For example, if the file \file{common.txt} contains
\begin{example}
/*N defs
    Definitions are described in Section 2.
N*/
\end{example}
then changing the \code{doctext} command to 
\begin{example}
doctext -latex -mpath ref/ref1 common.txt *.c
\end{example}
will cause each \code{.N defs} command to include the text ``Definitions
are described in Section 2.''

\node Making the man Pages Available,,,Top
\section{Making the man Pages Available}
In order to make the regular man pages available, a
directory tree containing directories \file{man}, \file{man/man1}, etc. (for
all of the man page extensions used) needs to be created.
In addition, a ``whatis'' database should be built within the primary man
directory so that \code{man -k topic}
will find the command or routines with \code{topic} in their description.
This is done with the program \file{/usr/etc/catman} or \file{makewhatis}.  If
either of these programs is not
available, the ``whatis'' database cannot be created.
The following C-shell code will create the database:
\begin{example}
unalias cd
if (-e /usr/etc/catman) then
    (cd man ; /usr/etc/catman -w -M . )
else if (-e `which makewhatis`) then
    (cd man ; makewhatis -M `pwd`)
endif
\end{example}

Some \code{catman} programs may behave erroneously; at least one system,
instead of using the \code{.SH NAME} entry, looks for \code{.SH NAME ...
.SH}.  What this means is that if the first text after the routine's
description is not a section title (i.e., a line ending with a colon), the
\code{catman} program will generate an incorrect whatis database.

It is easy to provide an X11 interface to the man pages by using \code{xman}
and the \code{MANPATH} variable.  Here is a simple shell script that provides
access to the man pages in \file{/home/me/man}:

\begin{example}
#! /bin/sh
MANPATH=/home/me/man
export MANPATH
xman -notopbox -helpfile /home/me/man/me.help "$@" &
\end{example}

In order to get \code{man} and \code{xman} to display the correct names for
the various sections (corresponding to the directories \file{man/man1},
\file{man/man2}, etc.), a file \file{man/mandesc} is required.  Here is the
\file{man/mandesc} for the PETSc package:

\begin{example}
r(r)Introduction
1(1)Sparse Matrix Routines
2(2)Vector Routines
3(3)Simplified Solvers
4(4)Iterative Methods
5(5)High Level Communications
6(6)Low Level Communications
7(7)System Calls
8(8)Miscellaneous
9(9)Domains and Grids
b(b)BLAS
x(x)X Window System Tools
n(n)Nonlinear Solvers
no default sections
\end{example}

Finally, \code{xman} displays a help page when it starts.  To change the file,
use the \code{-helpfile} argument and provide a simple text file (not a man)
page.  

\node Customizing the Output Format,,,Top
\section{Customizing the Output Format}
The text formating commands that \code{doctext} uses for its output are
controlled by several files.  For each output format (nroff, html, and latex)
there are two sets of definitions files.  The first set are the ones common to
all of the tools in the sowing package, and are in the directory
\file{share}. The files are
\begin{description}
\item[html.def]HTML commands
\item[latex.def]LaTeX commands
\item[nroff.def]Nroff (man) commands
\end{description}
In addition, the directory \file{share/doctext} contains special
versions of these same files.  \code{doctext} uses the files in
\file{share/doctext} by default (see the file
\file{sowing/src/doctext/docpath.h}).  You can override or add to these
commands with the \code{-defn filename} argument.  The environment variable
\code{DOCTEXT_PATH} may be used to point to a different set of definition
files.  

Small changes can be made by using the \code{-defn filename} switch to load a
file that redefines a few of the commands.  The file \file{htmlargtbl.def},
mentioned above for using the HTML table commands for arguments, is an
example.  Similarly, the file \file{latexargtbl.def} can be used to place the
argument definitions in a LaTeX \code{tabular} environment.  Using the
\code{-defn} argument to override a few definitions is usually preferable to
replacing all of the definitions.

\node The Doctext Commands,,,Customizing the Output Format
\subsection{The Doctext Commands}
\code{Doctext} uses named commands for all of its output.  For each of these
commands, there is a definition that tells \code{doctext} what to do
for each of these.  For example, when \code{doctext} outputs a section
(from a line ending in a \code{:}), \code{doctext} uses the
\code{section} command.  The default definition for this when
generating HTML is
\begin{example}
  section    %n<H2>%1</H2>%n
\end{example}
This instructs \code{doctext} to output a newline if it isn't already
at the start of a line (the \code{%n}); followed by standard HTML for
a level 2 heading, followed by another newline.  The \code{%1} in the
argument string stands for the first argument to the \code{section}
command.  To change the output style, you need only replace the
command definitions.  The next two sections detail the commands and
the command language.

\node Commands,,,Customizing the Output Format
\subsection{Commands}
\label{sec:outputcommands}
\begin{description}
\item[blank]Generate a blank space.  This is usually just a blank space.
\item[bof]Beginning of file command.  This is called when a new file
is created.  There is one argument: the name of the directory.
\item[bop]Beginning of page command.  This is called when starting a
new manual page.
\item[definition]There is one argument, the one-line description of the
  routine. Used only by \code{doc2lt}.
\item[em]Begin emphasis (or italic) style text.  See the command
syntax for the discussion of how font modes like this are exited.
\item[em_dash]Generate an em dash (---).
\item[end_par]Generate an end-of-paragraph.
\item[eof]End of file command.  This is called just before
\code{doctext} closes a file.
\item[eop]End of page command.  This is called at the end of a man
page.
\item[key]There is one argument, the name of the routine that a
man page is being generated for.  Used only by \code{doc2lt}
\item[linebreak]Generate a forced line break.
\item[location]Show the location of the file the generated this man
page.  The argument is the filename.
\item[mantitle]Generate the man page title.  This takes four string
arguments.  They are, in order
    \begin{description}
    \item[name]Name of the routine
    \item[level]Man page section (e.g., \code{3})
    \item[date]Date of last change to file
    \item[heading]man section name (from the \code{-heading} argument
    to \code{doctext}).
    \end{description}
\item[picture]This command takes the name of a file as argument and
will be used in future versions of \code{doctext} to allow the
inclusion of graphics in man pages.
\item[postamble]Used only by \code{doc2lt}
\item[preamble]Used only by \code{doc2lt}
\item[rm]Change font style to roman.
\item[section]Generate a man page section header.  There are two
arguments.  The string argument is the name of the section; the
integer argument is the level of the section.  The default section
level is two.
\item[synopsis]Generate the \code{Synopsis} section of the man page.
The argument is the synopsis text.  Used only by \code{doc2lt}.
\item[tt]Change font style to fixed width.
\end{description}
In addition to the above commands, there are a number of commands that
come in pairs, with a start (\code{s_xxx}) and an end (\code{e_xxx})
term.  
\begin{description}
\item[s_arg]Begin an argument (\code{.} format command).
\item[s_defn]Begin the definition of an argument.
\item[s_arg_list]Begin a list of argument (the \code{+} format
command).  This doesn't include the argument itself; that is begun
with the \code{s_arg_inlist} command.
\item[s_arg_inlist]Begin an argument within an argument list (started
with \code{s_arg_list}).
\item[s_defn_inlist]Begin the definition of an argument in an argument
list.
\item[s_caption]Begin the text for a caption.  This is intended for
use with the \code{picture} command, and is not supported.
\item[s_doctext_verbatim]Begin the single-line verbatim mode (the
\code{$} command).
\item[s_synopsis]Begin the synopsis section.
\item[s_verbatim]Begin a multiline verbatim section (\code{.vb}).
\end{description}

\node The Definition Command Language,,,Customizing the Output Format
\subsection{The Definition Command Language}
The command language for the definitions is very simple.  Each line
has the form
\begin{example}
command-name replacement-text
\end{example}
The \code{command-name} is any of the names in
Section\tie\ref{sec:outputcommands}. 
Comment lines may be included anywhere and start with \code{#}.
The \code{replacement-text} is almost literal text, except for
\code{%} escapes.  These escapes are almost all single letters; for
example, \code{%n} and \code]Output the \code{%} character
\item[1]Insert the first string argument
\item[2]Insert the second string argument
\item[3]Insert the third string argument
\item[4]Insert the fourth string argument
\item[i]Insert the first integer argument
\item[n]Insert a newline only if not at an newline
\item[u1]Insert the first string argument, but in uppercase.
Similarly for \code{u2} through \code{u4}.
\item[p]Insert end-of-paragraph string only if not at an
end-of-paragraph.
\item[f]Insert the end of font string.  This string is defined by the
\code{%e=} command.
\item[N="..."]Replace newlines with the string given in quotes.  The
string can be 32 characters or less.
\item[e="..."]Define the end of font string.  The string can be 32
characters or less.
\item[m="..."]Define the output mode.  This is a string that is
understood by the particular output format.  For example, the LaTeX
output routines understand the mode \code{verbatim} and use that to
change how characters are output.
\item[a0="..."]Define the value of string register zero.  The string may be up
  to 63 characters long.  There are 10 registers, commands \code{a1} through
  \code{a9} access the other nine.  The value of \code{a0} is set to the name
  of the function or macro by \code{doctext}, and normally should not be
  changed. 
\item[r1]Insert the value of string register zero.  Similarly for \code{r1}
  through \code{r9}.
\end{description}
To continue \code{replacement-text} to another line, end the line with
the \bw\ character.  A single space at the end of a line may be
represented with \bw<blank>.

Sometimes you will not want to replace a command; instead, you will
want to insert your replacement text either before or after the
standard text.  You can do this by placing a \code{+} either before or
after the \code{command-name}.  An example of this is in
Section\tie\ref{sec:alternatehtml}. 

\node Examples of Command Definition,,,Customizing the Output Format
\subsection{Examples of Command Definition}
This first example shows how to place the argument names and
definitions into a table when generating HTML output.  This works only
with the argument list commands (start a list with \code{+} and end it
with \code{-}).  Here are the definitions:
\begin{example}
# You can get this by specifying -defn htmlargtbl.def to doctext.
s_arg_list    %n<TABLE BORDER=0>
s_arg_inlist  %n<TR><TD WIDTH=20></TD><TD ALIGN=LEFT VALIGN=TOP><B>
e_arg_inlist  </B></TD>
s_defn_inlist <TD VALIGN=TOP>
# The br is used to break the line in the event that the browser doesn't
# support tables.  It is better than nothing.
e_defn_inlist <BR></TD></TR>
e_arg_list    </TABLE>%n
\end{example}
This example uses only the \code{%n} escape.  The definitions are
ordered as they would be used by \code{doctext}.

This second example makes \code{tt} text one size larger when
generating HTML:
\begin{example}
tt     %f<TT><FONT SIZE=+1>%e="</FONT></TT>"
\end{example}
The \code{%f} command is used to terminate any previous font style
command.  The \code{%e} command defines how to terminate this font
style.

\node Installing doctext,,,Top
\section{Installing doctext}
The \code{doctext} program is part of the a sowing package.
(An earlier version was part of the PETSc package of tools for
scientific computing.)
The program is available from
\URL{ftp://ftp.mcs.anl.gov/pub/sowing/sowing.tar.gz}. 
Directories and files such as \file{refman.sty} 
referred to in this manual are provided
in this implementation and may be found in the installation.
To build, 
\begin{example}
    gunzip -c sowing.tar.gz | tar xf -
    cd sowing
    ./configure 
    make
\end{example}
The \code{doctext} program is in the \file{sowing/src/doctext} directory.

\code{doctext} requires a C++ compiler; most testing has been done with
\code{gcc}.  

Please send any comments to \code{gropp@mcs.anl.gov}.

\c Currently on the list of requested features are:
\c preserve spaces in <PRE> in HTML (nearly impossible)
\c comment on erroneous use of &nbsp;?
\c see  http://www.hut.fi/~jkorpela/HTML3.2 for why LISTING not used

\addcontentsline{toc}{section}{Acknowledgment}
\section*{Acknowledgment}
The author thanks Lois Curfman McInnes and Barry Smith for their careful
reading and vigorous use of the \code{doctext} manual and program and
Ewing Lusk for valuable suggestions about additional functionality.

\begin{tex}
\addcontentsline{toc}{section}{References}
\bibliography{tools}
\bibliographystyle{plain}
\end{tex}

\end{document}
