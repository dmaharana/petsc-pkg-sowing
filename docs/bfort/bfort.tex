\documentclass[twoside]{../texlocal/linfoem}
%\usepackage[dvipdfm]{hyperref}
\usepackage{epsf,../texlocal/refman2,../texlocal/tpage}
\usepackage{../texlocal/fileinclude}
\usepackage{../texlocal/urlsimple}
\c -*-latexinfo-*-
\c \anltmtrue
\anltmfalse
\c\pagestyle{empty}
\c
\c We really need a ``usage'' macro for the routines and macros
\def\usage#1{}
\c
\c the definition of \file should be changed so that it adds an
\c \penalty 1000\hskip 0pt +1fil \penalty -10000
\c before the actual text (allowing a line that the file name would stick
\c out of to instead be raggedright, a not perfect but better approach).
\c
\c
\c See /home/beumer/GROPP/TMS/bfort.tex
\c
\let\DOESupport\DefaultSupport

\def\tie{\nobreak\ \nobreak}
\def\bw{{\tt\char`\\}}

\textheight=9in
\textwidth=6.5in
\oddsidemargin=0in
\evensidemargin=0in
\topmargin=-0.5in
\hsize=\textwidth

\begin{document}
\bibliographystyle{plain}
\newindex{fn}

\begin{ifinfo}
\c %**start of header
\title{Users Manual for bfort:\\
Producing Fortran Interfaces to C Source Code}
\author{William Gropp\\
Mathematics and Computer Science Division\\
Argonne National Laboratory}

\date{\today}

\c %**end of header

\maketitle
\end{ifinfo}

\begin{iftex}
\ANLTMTitle{Users Manual for bfort:\\
Producing Fortran Interfaces to C Source Code}{\em William Gropp\\
Mathematics and Computer Science Division}{208}{March 1995}

\end{iftex}

\clearpage

\c \vskip 0pt plus 1filll
\c Copyright \copyright{} 1995 Argonne National Laboratory
\c
\c \clearpage
\c \pagenumbering{roman}
\c \setcounter{page}{3}
\c \pagestyle{plain}
\c \tableofcontents
\c \clearpage

\pagenumbering{arabic}
\pagestyle{plain}
\thispagestyle{plain}

\setfilename{../bfort.info}

\node Top,        Introduction, (dir), (dir)
\c    nodename,   next,          previous, up

\c ---------------------------------------------------------------------------
\begin{iftex}

\pagenumbering{arabic}
\setcounter{page}{1}
\c \thispagestyle{empty}

\begin{center}
{\large\bf Users Manual for bfort:\\
Producing Fortran Interfaces to C Source Code\\}
\vspace{.2in}
by \\
\vspace{.2in}
{\it
William Gropp\\
}
\end{center}

\vspace{.2in}

\addcontentsline{toc}{section}{Abstract}
\begin{abstract}
\noindent 
In many applications, the most natural computer language to write in may 
be different from the most natural language to provide a library in.  For
example, many scientific computing applications are written in Fortran, while
many software libraries---particularly those dealing with complicated data
structures or dynamic memory management---are written in C.  
Providing an interface so that Fortran programs can call routines written in C
can be a tedious and error-prone process.  We describe here a tool that 
automatically generates a Fortran-callable wrapper for routines written
in C, using only a small, structured comment and the declaration of the
routine in C.  This tool has been used on two large software packages, PETSc
and the MPICH implementation of MPI.
\end{abstract}
\end{iftex}


\clearpage
\c ---------------------------------------------------------------------------



\c ---------------------------------------------------------------------------

\node Introduction,bfort,Top,Top
\section{Introduction}
\label{chap-intro}
The program described in this report is intended to help you create
interfaces to routines written in C that are callable from Fortran.
Specifically, the program \code{bfort} takes files containing C programs and
generates a new file for each routine that contains a C routine, callable from
Fortran, that will correctly call the original C routine.  The interface,
often called a wrapper routine or just wrapper, handles the issues of Fortran
routine name, variables passed by value instead of reference, and values 
generated and returned by the C library that are pointers to opaque objects.

\section{Getting Started}
Using \code{bfort} is easy.  For example, the command
\begin{example}
bfort foo.c
\end{example}
will generate Fortran interfaces for all of the commented routines in the file
\file{foo.c}.
But before this will do you any good, you will need to add some structured
comments to your file.

\section{Structured Comments}
The program \code{bfort} searches for C comments of the form \code{/*}{\em c}
\code{...} 
{\em c}\code{*/}, where {\em c} is a single character indicating the type of
documentation.  The available types include  \code{@} for
routines and \code{M} for C macros.
These structured comments can be used by \code{doctext} \cite{doctext} to
automatically generate manual pages and may containing additional information.
In all cases, the structured comment has the form
\begin{example}
/*@ 
    name - short description

    heading 1:

    heading 2:

    ...
@*/
\end{example}
The structured comment for a routine must immediately precede the declaration
of the routine (currently only K\&R-style declarations; ANSI-style prototypes
will be supported in a later release).
Figure \ref{fig-example} shows the structured comment and the routine being
documented.

\subsection{C Routines}
C routines are indicated by the structured comment \code{/*@ ... @*/}.
The program \code{bfort} uses the name in the first line (\code{Add} in this
example) to 
generate the name of the routine for which a wrapper is being provided.

\begin{figure}
\begin{example}
/*@
   Add - Add two values

   Parameters:
.  a1,a2 - Values to add
@*/
int Add( a1, a2 )
int  a1, a2;
\{
return a1 + s2;
\}
\end{example}
\caption{C code for an {\tt Add} routine with \protect\code{bfort}-style
structured comment}\label{fig-example} 
\end{figure}

\subsection{C Macros}
C macros are indicated by the structured comment \code{/*M ... M*/}.
Unlike the case of C routines, macro definitions do not provide any
information on the types of the arguments.
Thus, the body of a structured comment
for a C macro 
should include a {\em synopsis} section, containing a declaration of the
macro as if it were a C routine.  For example, if the \code{Add} example were
implemented as a macro, the structured comment for it would look like
\begin{example}
/*M
   Add - Add two values

   Parameters:
.  a1,a2 - Values to add

   Synopsis:
   int Add( a1, a2 )
   int a1, a2;
M*/
\end{example}
It is important that the word {\em Synopsis} be used; \code{bfort} and
related programs (\code{doctext} and \code{doc2lt}) use this name to find the
C-like declaration for the macro. 

\subsection{Indicating Special Limitations}
Two modifiers to the structured comments indicate special 
behavior of the function.  The modifiers must come after the character that
indicates a routine, macro, or documentation.
The modified \code{C} indicates that this routine is available only in C (and
not from Fortran).  If this modifier is present, \code{bfort} will not
generate a wrapper for the routine.  For example, a routine that returns a
pointer to memory cannot be called from standard Fortran 77, so its
structured comment should be 
\begin{example}
/*@C
  ....
@*/
\end{example}

The other modifier is \code{X}; this indicates that the routine requires the
X11 Window System.  This is used by \code{bfort} to provide an \code{#ifdef}
around the call in case X11 include files are not available.

The modifiers \code{C} and \code{X} may be used together and may be specified
in either order (i.e., \code{CX} or \code{XC}).

\subsection{Indicating Include Files}
It is often very important to indicate which include files need to be used with
a particular routine.  For this purpose, you may add a special structured
comment of the form \code{/*I}{\em include-file-name}\code{I*/}.
For example, to indicate that the routine requires that \file{<sys/time.h>}
has been included, use
\begin{example}
#include <sys/time.h>          /*I <sys/time.h> I*/
\end{example}
in the C file.
A user-include can be specified as
\begin{example}
#include "system/nreg.h"      /*I "system/nreg.h" I*/
\end{example}
This approach of putting the structured include comment on the same line as
the include of the file ensures that if the source file is changed by removing
the include, the documentation and Fortran wrappers will reflect that change.

\section{Command Line Arguments}
To use \code{bfort}, you need only to give it the name of the files
to process.  For example, to process every \code{.c} and \code{.h} file in the
current directory, use
\begin{example}
bfort *.[ch]
\end{example}
Command-line options to \code{bfort} allow you to change the details of
how 
\code{bfort} generates output.

A complete list of the command line options follows.  Some of these will
be used often (e.g., \code{-anyname}); others are
needed only in special cases (e.g., \code{-ferr}).

\vspace{5pt}
The following option controls the location of the output file:
\begin{description}
\item[-dir name]Directory for output file
\end{description}
In all cases, the name of the output file is the name of the input file with
an \code{f} suffix.  For example, if the input file is \file{foo.c}, the file
containing the generated wrappers is \file{foof.c}.

\vspace{5pt}
The following options control the kind of messages that \code{bfort} produces
about the 
generated interfaces. 
\begin{description}
\item[-nomsgs]Do not generate messages for routines that cannot be converted
              to Fortran.
\item[-nofort]Generate messages for all routines/macros without a Fortran
              counterpart.
\end{description}

\vspace{5pt}
The following options provide special control over the form of the interface
and the 
generated file.
\begin{description}
\item[-anyname]Generate a single wrapper that can handle the three most common
               cases: trailing underscore, no underscore, and all caps.  
               The choice is based on whether one of the following macro names
               is defined.
\begin{description}
\item[FORTRANCAPS]Names are upper case with no trailing underscore.
\item[FORTRANUNDERSCORE]Names are lower case with trailing underscore.
\item[FORTRANDOUBLEUNDERSCORE]Names are lower case, with {\em two} trailing
      underscores.  This is needed when some versions of \code{f2c} are 
      used to generate C for Fortran routines.  Note that \code{f2c} uses two
      underscores {\em only} when the name already contains an underscore
      (on at least one FreeBSD system that uses \code{f2c}).
      To handle this case, the generated code contains the second
      underscore only when the name already contains one.
\end{description}
In addition, if \code{-mpi} is used, the MPI profiling names are 
also generated, surrounded by \code{MPI_BUILD_PROFILING}.

\item[-ferr]Fortran versions return the value of the routine as the last
              argument (an integer).  This is used in MPI and is a 
              common approach for handling error returns.

\item[-I name]Give the name of a file that contains \code{#include}
statements that are necessary to compile the wrapper.

\item[-mapptr]Generate special code to convert Fortran integers to and from
pointers used by the C routines.  The special code is used only if the macro
\code{POINTER_64_BITS} is defined.  
It is also used to determine whether pointers are too long to
fit in a 32-bit Fortran integer.  (You have to insert a call to 
\code{__RmPointer( pointer )} into the routines that destroy the pointer.)
The routines for managing the pointers are in \file{ptrcvt.c}.

\item[-mnative]Multiple indirects (\code{int ***} are native datatypes; that
is, there is no coercion to the basic type).

\item[-mpi]Recognize special MPI \cite{mpi-final} datatypes (some MPI
datatypes are pointers by definition). 

\item[-ptrprefix name]Change the prefix for names of functions to convert
to/from pointers.   The default value of \code{name} is \code{__}.

\item[-voidisptr]Consider \code{void *} as a pointer to a structure.
\end{description}

\begin{description}
\item[-ansiheader]Generate ANSI-C style headers instead of Fortran interfaces.
  This is useful in creating ANSI prototypes  without converting the code to 
  ANSI prototype form.
  These use a trick to provide both ANSI and non-ANSI prototypes.
  The declarations are wrapped in \code{ANSI_ARGS}, the definition of which 
  should be
\begin{verbatim}
#ifdef ANSI_ARG
#undef ANSI_ARG
#endif
#ifdef __STDC__
#define ANSI_ARGS(a) a
#else
#define ANSI_ARGS(a) ()
#endif
\end{verbatim}
\end{description}

After the command-line arguments come the names of the files for which Fortran
interfaces are to be constructed.

\section{Examples}
This section shows the code generated by \code{bfort} with various
command-line switches.

\vspace{5pt}
The command \code{bfort add.c} produces
\begin{verbatim}
/* add.c */
/* Fortran interface file for sun4 */
 int  add_( a1, a2)
int*a1,*a2;
{
return Add(*a1,*a2);
}
\end{verbatim}

\vspace{5pt}
The command \code{bfort -anyname add.c} produces
\begin{verbatim}
/* add.c */
/* Fortran interface file */
#ifdef FORTRANCAPS
#define add_ ADD
#elif !defined(FORTRANUNDERSCORE) && !defined(FORTRANDOUBLEUNDERSCORE)
#define add_ add
#endif
 int  add_( a1, a2)
int*a1,*a2;
{
return Add(*a1,*a2);
}
\end{verbatim}

\vspace{5pt}
The command \code{bfort -ansiheader foo.c} produces
\begin{verbatim}
/* add.c */
extern int  Add ANSI_ARGS((int, int ));
\end{verbatim}

For a more sophisticated example, here is the result of \code{bfort -ferr -mpi
-mnative -mapptr -ptrprefix MPIR_ -anyname -I pubinc send.c} for the MPI
routine \code{MPI_Send} (implemented in the file \code{send.c}, from the MPICH
implementation).  The file \file{pubinc} contains the single line
\code{#include "mpiimpl.h"}. 

\begin{verbatim}
/* send.c */
/* Fortran interface file */
#include "mpiimpl.h"

#ifdef POINTER_64_BITS
extern void *MPIR_ToPointer();
extern int MPIR_FromPointer();
extern void MPIR_RmPointer();
#else
#define MPIR_ToPointer(a) (a)
#define MPIR_FromPointer(a) (int)(a)
#define MPIR_RmPointer(a)
#endif

#ifdef MPI_BUILD_PROFILING
#ifdef FORTRANCAPS
#define mpi_send_ PMPI_SEND
#elif defined(FORTRANDOUBLEUNDERSCORE)
#define mpi_send_ pmpi_send__
#elif !defined(FORTRANUNDERSCORE)
#define mpi_send_ pmpi_send
#else
#define mpi_send_ pmpi_send_
#endif
#else
#ifdef FORTRANCAPS
#define mpi_send_ MPI_SEND
#elif defined(FORTRANDOUBLEUNDERSCORE)
#define mpi_send_ mpi_send__
#elif !defined(FORTRANUNDERSCORE)
#define mpi_send_ mpi_send
#endif
#endif

 void mpi_send_( buf, count, datatype, dest, tag, comm, __ierr )
void             *buf;
int*count,*dest,*tag;
MPI_Datatype     datatype;
MPI_Comm         comm;
int *__ierr;
{
*__ierr = MPI_Send(buf,*count,
        (MPI_Datatype)MPIR_ToPointer( *(int*)(datatype) ),*dest,*tag,
        (MPI_Comm)MPIR_ToPointer( *(int*)(comm) ));
}
\end{verbatim}

Note that the \code{-mapptr} option has caused the generated code to call
routines to convert the integers in Fortran to valid pointers.  The option
\code{-ptrprefix} changed the names of the routines to be
\code{MPIR_ToPointer} and \code{MPIR_FromPointer}.  The option \code{-mpi}
informed \code{bfort} that \code{MPI_Datatype} and \code{MPI_Comm} were
pointers rather than nonpointers.  The option \code{-ferr} converted a
routine \code{MPI_Send} that returns an error code to a Fortran subroutine
that returns the error code in the last argument.  

\section{Installing bfort}
The \code{bfort} program is part of the PETSc package of tools for
scientific computing, but can be installed without installing all of PETSc.
The program is available from \file{info.mcs.anl.gov} in
\file{pub/petsc/textpgm.tar.Z}.  Additional information is available through
the World Wide Web at \URL{http://www.mcs.anl.gov/petsc}.

Please send any comments to \code{gropp@mcs.anl.gov}.

\section*{Acknowledgment}
\addcontentsline{toc}{section}{Acknowledgment}
The author thanks Lois Curfman McInnes and Barry Smith for their careful
reading and vigorous use of the \code{bfort} manual and program.

\begin{tex}
\addcontentsline{toc}{section}{References}
\bibliography{../tools}
\end{tex}

\end{document}
