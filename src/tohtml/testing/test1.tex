\documentstyle{article}
\newcommand{\commentOut}[1]{{}}
\newcommand{\FooBar}{{}}

\def\MPI/{MPI}

\begin{document}
\section{A sample}\label{Sample-doc}

Pi = \( \pi \).

\begin{figure}[htbp]
    \centerline{Foo!}
    \caption{Integrating to find the value of \protect\( \pi \protect\)}
    \label{fig:pi}
\end{figure}

Here is \MPI/.
\commentOut{
\begin{verbatim}
           First Execution

    0             1               2
                        /-----  send
                recv <-/
broadcast     broadcast       broadcast
  send ---\
           \--> recv
\end{verbatim}

\begin{verbatim}
           Second Execution

   0              1               2
broadcast
  send ---\
           \-->  recv
               broadcast       broadcast
                           /---  send
                 recv <---/
\end{verbatim}
} % end commentOut -- NOTE that above was wrong!

Here is an enumeration
\begin{enumerate}
\item[] global max on integer and floating point data types
\item[] global min on integer and floating point data types
\item[] global sum on integer and floating point data types
\item[] global product on integer and floating point data types
\item[] global AND on logical and integer data types
\item[] global OR on logical and integer data types
\item[] global XOR on logical and integer data types
\item[] rank of process with maximum value
\item[] rank of process with minimum value
\item[] user defined (associative) operation
\item[] user defined (associative and commutative) operation
\end{enumerate}
which ended here.  And now, a displayed equation
\[
A x = n
\]
This is an inline \( \epsilon = \mu \) equation.  This is an inline equation
with only regular text \( a = ( 1,2 ) \), and this one has braces \( \{ (1,2),
(3,4)\} \).  Here are the same things, but using the dollar notation:

$$
A x = n
$$
This is an inline $ \epsilon = \mu $ equation.  This is an inline equation
with only regular text $ a = ( 1,2 ) $, and this one has braces $ \{ (1,2),
(3,4)\} $.


This is an inline equation with dollars: {\sf $<$type$>$}.

\begin{eqnarray}
a & b & c \\
d & e & f \\
\end{eqnarray}
\begin{verbatim}
foo
\end{verbatim}
%\end{document}
this is a test

\begin{figure}
some stuff
\caption{This is a test}\label{fig-foo}
\end{figure}

This refers to Figure \ref{fig-foo}, which may be found in \ref{Sample-doc}.
\end{document}
