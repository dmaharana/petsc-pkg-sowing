\documentclass[11pt,epsf,/home/gropp/HPCCP/formats/handpage,../../tex/hyper]{article}
% Use ../../tex/hyper.sty for hypertext definitions
\addtolength{\textheight}{2.5in}
\pagestyle{empty}
\begin{document}
\pagestyle{empty}
\title{SLES --- State-of-the-Art Linear Systems Solvers
\thanks{Mathematics and Computer Science Division,
Argonne National Laboratory,
Argonne, IL 60439-4801.
This work was supported by the Applied Mathematical
Sciences subprogram of the Office of Energy Research, U. S. Department of
Energy, under Contract W-31-109-Eng-38.}}

\date{}
\maketitle

\section*{Description}

\code{SLES} is a package of routines that 
provides programmers with an easy to use, efficient, and
extensible access to state-of-the-art methods, both direct and iterative,
for solving systems of linear equations.
The design of \code{SLES} allows new methods and implementations to be 
added to the libraries and made available to a user without any change 
to the user's programs.  

\code{SLES} is organized into several layers.  
The simplest layer allows the programmer to specify the matrix and solve
a linear system.
Calling sequences are short and simple.
More advanced users can select from a wide variety of iterative methods
(for both symmetric and nonsymmetric problems), preconditioners, and
sparse direct methods.
A ``registry'' mechanism enables users to add iterative methods,
preconditioners, and sparse direct-ordering techniques.
User-specified matrix formats are supported, allowing programs to
use the data structures that are natural for them.  

The methods included with the package include sparse direct methods,
iterative preconditioners such as ILU with fill, ICC, and SSOR; and 
accelerators such as GMRES, BiCG-Stab, CG, and transpose-free QMR.

\section*{Computational Environment}
\code{SLES} is portable, simple to install and use, and 
efficient.
It is installed on most new systems as they become available.
Further, the routines are designed to be completely data-structure neutral.
\code{SLES} is available on the Sun, DEC, Silicon Graphics, and IBM
RS/6000 workstations; IBM SP; Intel iPSC/860, DELTA, and Paragon;
Convex; and Cray machines.

\section*{Availability}

The \code{SLES} package is freely available.
The complete distribution can be obtained by anonymous ftp from
info.mcs.anl.gov.  Take the file sles.tar.Z from the directory pub/pdetools.

\section*{Documentation}

The \code{SLES} distribution contains all source code, 
installation instructions, a \hcitea{sles-user-ref}{users guide}
in both ASCII text and latexinfo format,
\hcitea{sles-man-pages}{man pages}, 
and a collection of examples in both C and Fortran.  
The following WWW address contains additional information about the
package: \hcitea{petsc-web-page}{http://www.mcs.anl.gov/home/gropp/petsc.html}.

\contact{William Gropp}
\email{gropp@mcs.anl.gov}
\phone{(708) 252-4318}
\makeinfo
\end{document}
