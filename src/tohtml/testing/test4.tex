\documentstyle{article}
\begin{document}

\newtheorem{example}{Example}[section]

\section{First Section}
\begin{verbatim}
<program>( &argv, &argv );
\end{verbatim}
\subsection{Examples}
This line has a \& ampersand in it.
Examples in first section
\hskip 1.15truein foo, bar.
\def\indef#1{.#1\hangindent
7em.}
\indef{\hskip 1.15truein MPI is fun}

\section{Second Section}
\subsection{Examples}
Examples in second section
\begin{example}
  A simple example including an escaped percent (\%) in a code \code{\%}
\end{example}

\subsection{Even more}
\subsection{Examples}
Examples in second subsection of the second section

\begin{example}
  A simple example involving $math$.
\end{example}

Simple math
$ a + b $
Less simple (braces)
$ \{ a + b \}$
And with ldots
$ Foo( a + b + \ldots ) $
Even worse, with subscripts and superscripts 
$ A_1 + B^2 = C_{old} $

Here is a block of text that caused some problems (overwrites in the
math handling):

When manipulating a preconditioning matrix, $ A $ , BlockSolve95
internally works with a scaled and permuted matrix, $ \hat{A} = P
D^{-1/2} A D^{-1/2},$ where $ D $ is the diagonal of $ A $ , and $ P $ is a
permutation matrix determined by a graph coloring for efficient
parallel computation.  Thus, when solving a linear system, $ Ax=b $,
using ILU/ICC preconditioning and the matrix format {\tt MATMPIROWBS}
for {\em both} the linear system matrix and the preconditioning
matrix, one actually solves the scaled and permuted system $ \hat{A}
\hat{x} = \hat{b} $ , where $ \hat{x} = P D^{1/2} x $ and $\hat{b} = P
D^{-1/2} b$ . 


\end{document}
