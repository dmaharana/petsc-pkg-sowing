
\newenvironment{mainpts}{\begin{list}{$\bullet$}{\setlength{\leftmargin}{.30in}}}{\end{list}}
\newenvironment{secpts}{\begin{list}{--}{\setlength{\leftmargin}{.35in}}}{\end{list}}

\newif\ifpictures
\picturesfalse
%\picturestrue

\vtitle{High Performance Computing Research Facility}
\begin{center}
\logo
Bill Gropp \\
Rusty Lusk\\[1.5in]
\end{center}
%\vspace{2in}
\vfill
\ve

\vtts{Outline}
  \begin{mainpts}
    \item Hardware
    \item Software
    \item Projects
    \item Applications
    \item Plans
  \end{mainpts}
\ve

\vtts{The IBM SP at Argonne}
\begin{small}
\begin{mainpts}
  \item 128 Power-1 nodes
  \item SP2 switch
  \item 8 file servers
  \item 220 GB RAID
  \item Fiber Channel from some nodes to file servers
  \item 6 TB tape with robot
  \item Futures Lab:  networking, multimedia laboratory
  \item CAVE 3D immersive virtual reality environment
  \item HiPPI from file servers to RAID, tape, CAVE, Futures Lab
\end{mainpts}
\end{small}
\ve

\vtts{The IBM SP at Argonne}
\begin{small}
\centerline{\epsfxsize=5in\epsfbox{sp2config.eps}}
\end{small}
\ve

\vtts{System Software}
\begin{small}
\begin{mainpts}
  \item AIX 3.2.5  (HiPPI support in 4.1?)
  \item Parallel Operating Environment (POE)
  \item Assorted message-passing and other libraries
  \item Scheduler
  \item I/O Software
\end{mainpts}
\end{small}
\ve

\vtts{Message-Passing Libraries for the SP}
\begin{small}
Variety of message-passing libraries used
%Users use a variety of libraries, all of which use the switch efficiently.
%Most are extremely portable, which contributed to the speed with which the
%SP became useful.
\begin{mainpts}
  \item p4:  Argonne portable system
  \item Chameleon:  Argonne portable system
  \item MPI:  Multiple implementations of the Message-Passing Interface
    standard: 
    \begin{secpts}
      \item MPI-CH:  Argonne-Miss.~State freely available portable
implementation 
      \item MPI-F:  Yorktown implementation
      \item MPI-K(?):  Kingston product beta
    \end{secpts}
  \item Fortran-M:  Argonne extensions to Fortran
  \item CC++:  Concurrent version of C++
  \item MPL:  IBM's proprietary library, successor to EUI and EUI-H
\end{mainpts}
All use switch efficiently

Most are extremely portable and enabled rapid use of the SP
\URL{http://www.mcs.anl.gov/Projects/sp1/guide-r2/guide-r2.html}
\end{small}
\ve

\vtts{The Argonne Scheduler}
\begin{small}
The SP came without a batch scheduler appropriate for a production use
supercomputer.  Argonne has supplied one.
\begin{mainpts}
  \item Provides for both interactive and batch jobs
  \item Provides exclusive access to nodes, for peak performance
  \item Provides predictable job initiation and termination
  \item Provides friendly interface to keep users happy
  \item Provides node-hour accounting
\end{mainpts}
We are working with IBM to provide some of the same functionality in
LoadLeveler.  Meanwhile, other sites with large SP's have installed the
Argonne scheduler.

\URL{http://www.mcs.anl.gov/home/rayl/scheduler/scheduler.html}
\end{small}
\ve

\vtts{I/O Software}
\begin{small}
\begin{mainpts}
  \item PIOFS for parallel I/O from nodes to file servers

	Several groups investigating; still unstable

%/piofs is now back up and running.  Due to a problem with the 
%servers, allof the current data was lost.  I have also 
%installed the latest version, so hopefully we won't be running
%into these same problems.  
  \item UniTree for managing server-RAID-tape hierarchy

	\begin{secpts}
	\item 500 GB stored in 15K files
	
	\item CAVE has direct access to UniTree files

	\item Working well as archival file system

	\item Parallel file support through Chameleon (ANL) IO library 
	\end{secpts}

	
\end{mainpts}
\URL{http://www.mcs.anl.gov/Projects/unitree/unitree.html}
\end{small}
\ve

\vtts{Collaborative Projects with IBM}
\begin{small}
%A number of specific joint projects have been initiated, and are beginning to
%produce results.
\begin{mainpts}
  \item Experimental switch configuration (servers as nodes)
  \item Scalable I/O Initiative
  \item Message-passing libraries for high-performance applications
  \item Scalable Unix Tools
  \item TOE
  \item Ease of use
  \item Competitive analysis
  \item Multimedia
  \item Integrated thread and communication libraries
  \end{mainpts}
\end{small}
\ve

\vtts{Servers as Nodes}
% Gail hasn't seen this page yet

\begin{small}
Attaching our rs6000 970 I/O servers into the SP switch fabric was a success
and brought us the following advantages

\begin{mainpts}
\item  Allowed us to test and run Hippi, ATM, and fibrechannel on our SP
   without having to buy the SP2 Wide nodes.

\item  Directly connected the switch network fabric to Argonne's I/O subsystem 

\item  Users can now write directly from the SP into our I/O subsystem
   at 70 MB/sec.  The switch will allow us to reach higher numbers,
   however we are limited by our IBM 9570 disk array configuration and
   performance.  

%   I doubt you want to mention this, but so you know,  Unitree is set up
%   on only 2 of the 4 disk arrays.  The other 2 are being used 
%   the parallel file system.  Each array has only 1 hippi card in it.
%   I have found we can only drive this card at about 35 MB/sec.  We could
%   bring the other two arrays into Unitree to double our existing
%   performance, but it would be a major configuration change and we
%   would lose using 2 arrays for other uses. (piofs, sgi filesystem using
%   the new maxstrat code, or just using them for raw performance tests)

\item  install was relatively easy and the 970s worked immediately.  We 
   have seen no major problems with running tb2 switch cards in 
   a non-SP rs6000.
\end{mainpts}
\end{small}
\ve

\vtts{Scalable I/O Initiative}
\begin{small}
Multi-institution, multi-vendor, multi-agency
effort to attack the problem of high performance I/O for scalable parallel
systems

Working groups:\\
(1) Applications, (2) performance measurement, (3) compilers and languages, 
(4) operating
systems and file systems, and (5) integration and testbeds.

ANL contributions:
\begin{mainpts}
\item 128 node IBM SP testbed
\item several applications participating in SIO
\item PIOFS Beta test
\item SIO API work (e.g., Chameleon interface to Unitree)
\item languages and runtime system 
\end{mainpts}

\URL{http://www.ccsf.caltech.edu/~jpool/SIO/SIO.html}
\URL{http://www.mcs.anl.gov/home/gropp/scalable/integration/index.html}
\end{small}
\ve

\vtts{Message-passing libraries}
\begin{small}
\begin{mainpts}
\item Functionality

\begin{secpts}
\item ANL identified probe and recv\&call as necessary for many message-passing
applications

\item ANL provided test cases and beta test for both
\end{secpts}

\item Standards (MPI)
\begin{secpts}
\item ANL provided MPICH for IBM's MPIF implementation, and was an MPIF beta
site 

\item Developing proposals and prototypes for dynamic process management
\end{secpts}
\end{mainpts}

\URL{http://www.mcs.anl.gov/Projects/mpi/index.html}
\URL{http://www.mcs.anl.gov/Projects/mpi/mpi2/mpi2}
\end{small}
\ve

\vtts{Scalable Unix Tools}
\begin{small}
\begin{mainpts}
\item Inspired by visit to Kingston to test 128 SP1 in 1993.

\item Natural extensions of common Unix commands to parallel, distributed
memory 
systems.

\item {\tt fps} and its parallel counterpart, {\tt pfps}, provide a
way to manage the ``space'' of processes

\item Adopted as first project by the Ptools Constortium project (co-sponsored by IBM and Meiko)

\item An incompatible version for system managers provided by IBM with SP2
software (includes {\tt pfps}).
\end{mainpts}

\URL{http://www.llnl.gov/ptools/ptools.html}
\URL{http://www.mcs.anl.gov/home/lusk/ptools}
\end{small}
\ve

\vtts{TOE (Tragedy of Errors)}
\begin{small}
Goal: Present a friendly and inviting interface to the user by making error
messages meaningful or by preventing the error from occurring.

Examples\\
\begin{tiny}
\begin{mainpts}
\item 275 Occurrences
\begin{verbatim}
Message catalog not found during Catalog Initialization Routine.
Your NLSPATH is probably not set correctly. Message Catalog name = "pepoe.cat".
current NLSPATH = "/usr/lib/nls/msg/%L/%N:/usr/lib/nls/msg/prime/%N".
NLSPATH strings after expansion of variables = ""
If NLSPATH is set correctly, check LANG or LC_MESSAGES variables
\end{verbatim}

\item 4 Occurrences
\begin{verbatim}
D1<L3>: pm_respond: Input file ready!
\end{verbatim}

\item 158 Occurrences
\begin{verbatim}
INFO: 0031-656  I/O file STDOUT closed by task <task_number>
\end{verbatim}

\item 2 Occurrences
\begin{verbatim}
ERROR: 0031-619  A=80=FF=F8|: Killed
\end{verbatim}

\item 7 Occurrences
\begin{verbatim}
ERROR: 0031-619  : Error number 73
\end{verbatim}
\end{mainpts}
\end{tiny}

Error handling code in IBM's POE makes it easy to capture these messages.
\end{small}
\ve

\vtts{Ease of use}
Goal: Make parallel computers easier to use.
\ve

\vtts{Competitive Analysis}
Outstanding issue (resolved tomorrow?) to look at the state of the art in
parallel computing in
more depth (to avoid a repetition of the ``probe'' and ``recv\&call''
problems). 

\ve

\vtts{Multimedia}
Bob Olson
\ve

\vtts{Nexus: Integrating Messaging and Multithreading}

\begin{small}
\begin{mainpts}
\item Multithreaded runtime system designed as
  a compiler target for parallel languages

\item  Used by compilers for CC++, Fortran M, HPF/M.

\item  Integrates multithreading, asynchronous communication, multiple
  address spaces/process, dynamic process management, multiple
  communication protocols, and shared-memory constructs

\item  Argonne/Yorktown/Caltech project is investigating optimizations
  for SP2 and integration with MPI.
\end{mainpts}
\URL{http://www.mcs.anl.gov/nexus/}
\end{small}
\ve

\vtts{Multithreading and Scalable I/O}

\begin{small}
\begin{mainpts}
\item Builds on Nexus

\item Exploits multithreading as means of overlapping I/O with
  computation and communication in parallel codes
%    \begin{secpts}
%    \item Colocate computation and I/O processes
%    \item Asynchronous 2-phase I/O operations
%    \item Noncollective I/O operations
%    \end{secpts}

\item Three related research efforts
    \begin{secpts}
    \item Multithreaded version of PASSION (Syracuse)
    \item Multithreaded version of Jovian (Maryland)
    \item Parallel I/O library for chemistry (PNL)
    \end{secpts}

\item Challenges
    \begin{secpts}
    \item Efficient preemptive messaging on SP2 (needs AIX 4.1,
	and integrating of messaging with threading?)
    \item Ability to block threads on parallel file system
	operations
    \end{secpts}
\end{mainpts}
\URL{http://www.mcs.anl.gov/dbpp/}
\end{small}
\ve

% Most are in ~pieper/SP-REPORT and have -report after them.
\vtts{Applications}

\begin{small}
\begin{mainpts}
  \item Nuclear Physics
  \item Computational Biology
  \item Computational Chemistry
  \item Astrophysics
  \item Materials Science
  \item Electromagnetics
  \item Genetic Algorithms
  \item Superconductivity
  \item Tools and Math Libraries
  \item SP Information Retrieval
\end{mainpts}
\URL{http://www.mcs.anl.gov/Projects/sp1/report}
\end{small}
\ve

\vtts{Nuclear Physics}
\begin{small}
\begin{mainpts}
\item Goal:
Determine phenomenological nuclear
potentials that can be 
used in calculating results of experiments on nuclei and properties of the
crust of neutron stars.  

%\centerline{\epsfbox{nuclear.ps}}
\item Participants:
\begin{secpts}
\item Argonne
\item Univ of Illinois at Urbanna
\end{secpts}
\end{mainpts}
\URL{http://www.mcs.anl.gov/Projects/sp1/report/nuclear-intro.html}
\end{small}
\ve

\vtts{Computational Biology}
\begin{small}


\begin{mainpts}
\item Goal: Develop computational tools needed to effectively address major
problems of interest to biologists

\item Computational Enzymology

Study the reaction pathways for an important enzyme (first step toward
rational drug design)

\URL{http://www.mcs.anl.gov/Projects/sp1/report/bash1.html}
\item Accurate
Phylogenetic Framework for
Biology 

Generate a phylogenetic tree based on the Ribosomal Database Project (vital in
understanding evolution of life)

\URL{http://www.mcs.anl.gov/Projects/sp1/report/overbeek.html}

\end{mainpts}

\end{small}
\ve

\vtts{Computational Chemistry}
\begin{small}
\begin{mainpts}
\item Goal: Develop algorithms and tools needed to exploit teraflops computers
in modeling macroscopic processes of the environment

\item Quantum dynamics 

Computation of the detailed quantum dynamics of a scattering event for $OH + CO
\rightarrow H + CO_2$.

\URL{http://www.mcs.anl.gov/Projects/sp1/report/GRAY/text.html}
\item Parallel CHARMM

Parallelization of widely used macromolecular system simulator

Joint work with PNL and five inductrial firms

\URL{http://www.mcs.anl.gov/Projects/sp1/report/shin2/shin2.html}
\end{mainpts}
\end{small}
\ve

\vtts{Convective penetration in stellar interiors}
%\ifpictures
%\centerline{\epsfxsize=5in\epsfbox{../SP1/1800.cps}}
%\else
%\vskip 5in
%\fi

\begin{small}
\begin{mainpts}
\item Goal: Simulate turbulent convection, which plays central role in dynamic
activity of stars, planets, and supernovae

\item Apporach: Develop parallel code using higher-order Gudunov method and
nonlinear multigrid  elliptic solver

\item Participants:
\begin{secpts}
\item Argonne
\item Univ. of Chicago
\end{secpts}


%(Thanks to Andrea Malagoli, Anshu Dubey, and Fausto Cattaneo, University of
%Chicago ({\tt malagoli@liturchi.uchicago.edu})
\end{mainpts}

\URL{http://www.mcs.anl.gov/Projects/sp1/report/MALAGOLI/mal.sp1_report_94.html}

\end{small}
\ve

\vtts{Materials Science}
\begin{small}
\begin{mainpts}
\item Goal: Develop theory to explain nature of glasses, a first stop in
predicting mechanical and thermal properties

\item Develop large dynamic models of glasses

\begin{secpts}
\item allow in-depth study
\item provide high-accuracy data to evaluate theories
\end{secpts}

\item Participants:\\
Univ of Illinois at Urbanna
\end{mainpts}
\URL{http://www.mcs.anl.gov/Projects/sp1/report/kieffer-report.html}
\end{small}
\ve

\vtts{Electromagnetics}
\begin{small}
\begin{mainpts}
\item Widely used in industrial applications such as MRI and semiconductors
\item Current limitations
\begin{secpts}
\item accuracy
\item problem size
\item solution time
\end{secpts}
\item Approach
\begin{secpts}
\item integral equations
\item high-performance computers
\end{secpts}
\item Results:
\begin{secpts}
\item parallel CORAL code competitive with traditional methods
\end{secpts}
%\item Use CAVE for 3D visualization
\item Participants
\begin{secpts}
\item Argonne 
\item Tampere University
\end{secpts}
\end{mainpts}
\URL{http://www.mcs.anl.gov/Projects/sp1/report/levine-report.html}
\end{small}
\ve

\vtts{Genetic Algorithms}
\begin{small}
\begin{mainpts}
\item First application of genetic algorithms to the set  partitioning problem
(SPP), particularly the airline crew scheduling problem.

\item Significant conclusions:
\begin{secpts}
\item     For such an important
    real-world problem with significant economic implications, {\em  genetic
algorithms are a viable solution methodology}.
\item     On up to 128 processors, {\em GAs scale very well} and show great
promise on larger MPP     systems.
\item      GAs can be
used on {\em much longer strings} than previously     known, opening the way
for 
additional applications of GA technology in the future.  
\item      A genetic algorithm hybridized with a hill-climbing heuristic
outperformed other methods
\end{secpts}
\end{mainpts}

\URL{http://www.mcs.anl.gov/Projects/sp1/report/levine-genetic.html}
\end{small}
\ve

\vtts{Superconductivity}
\begin{small}
\begin{mainpts}
\item  Birth of a Vortex

One of the main obstacles in the development of practical high-temperature
superconducting 
materials is dissipation. Dissipation is caused by the motion of magnetic flux
quanta called {\em vortices}.
Numerical simulation allows close-up views of the entry and subsequent
diffusion of vortices into a superconductor. 
 
\URL{http://www.mcs.anl.gov/Projects/sp1/report/super.html}

\item Twin Boundaries -

Natural defect that can improve the
current-carrying capabilities of high-temperature superconductors by
inhibiting flux motion. 
Simulation allows detailed view of the actual vortex dynamics in the presence
of twin boundaries.  

\URL{http://www.mcs.anl.gov/Projects/sp1/report/twin.html}
\end{mainpts}
\end{small}
\ve

\vtts{Tools and Math Libraries}
\begin{small}
\begin{mainpts}
\item Parallel Software for Large-Scale Eigenproblems

\URL{http://www.mcs.anl.gov/Projects/sp1/report/bischof-report/bischof-report.html}

\item Multithreading and Scalable
Input/Output
\URL{http://www.mcs.anl.gov/Projects/sp1/report/ian.html}

\item Nexus
\URL{http://www.mcs.anl.gov/Projects/sp1/report/foster.html}

\end{mainpts}
\end{small}
\ve

\vtts{SP Information Retrieval}
\begin{small}
\begin{mainpts}
\item Petabyte Access and Storage
Solutions (PASS) seeks to develop {\em scalable
solutions to the problems of data retrieval} and analysis.
\item Demonstration project with Fermilab and Argonne
\end{mainpts}

\URL{http://www.mcs.anl.gov/Projects/sp1/report/may-report.html}
\end{small}
\ve

\vtts{Gap}
Plans go here (Rusty?)
\ve

\vtts{Summary}
\begin{small}
\begin{mainpts}
  \item IBM SP at Argonne a big success
  \item Collaboration working well
  \item More successes at hand
\end{mainpts}
\end{small}
\ve
