\documentclass{article}
\begin{document}
% ---------------------------------------------------------------------------
% Specify this next section directly in manual_tex.tex and manual.tex due 
% to formatting differences in latex and html documents.
% ---------------------------------------------------------------------------

\noindent {\bf Getting Information on PETSc:} 

\medskip

\noindent {\bf On-line:}
\begin{list}{$\bullet$}
\item Manual pages on all routines including example usage
\begin{list}{$\bullet$}
   \item {\tt docs/manualpages/index.html} in the distribution or 
   \item {\tt http://www.mcs.anl.gov/petsc/docs/manualpages/index.html}
\end{list}
\item Troubleshooting
\begin{list}{$\bullet$}
   \item {\tt docs/troubleshooting.html} in the distribution or
   \item {\tt http://www.mcs.anl.gov/petsc/docs/troubleshooting.html}
\end{list}
\end{list}

\noindent {\bf In this manual:}
\begin{list}{$\bullet$}
{
\setlength{\itemsep}{-.02in} 
\setlength{\topsep}{0in} 
\setlength{\partopsep}{0in}
}
\item Basic introduction: \pageref{sec:gettingstarted}
\item Assembling vectors and matrices: \pageref{sec:vecintro} and \pageref{sec:matintro}
\item Linear solvers: \pageref{ch:sles}
\item Nonlinear solvers: \pageref{chapter:snes}
\item Timestepping (ODE) solvers: \pageref{chapter:ts}
% \item List of all routines and PETSc data types, page \pageref{routines}
% \item Function index, page \pageref{sec:findex}
% \item Subject index, page \pageref{sec:sindex}.
\item Index: \pageref{sec:index}.
\end{list}
\end{document}
