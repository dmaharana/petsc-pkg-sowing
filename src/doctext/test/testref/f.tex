\startmanpage
\mantitle{fancy}{tex}{2/10/2000}
\manname{fancy}
--- Shows all fancy formatting 
\subhead{Synopsis}
\startvb\begin{verbatim}
#include "tools.h"
#include "system/system.h"
#include <sys/time.h> 
#include <sys/types.h> 
void f1( int a, char *b )
\end{verbatim}
\endvb

\subhead{Input Parameters}
\startarg{x, y }{Coords
}
\startarg{a }{the {\em really} important value
}
\par
\subhead{Output Parameters}
\startarg{p1 }{pointer {\em to a}. Is this here?
}
\par
\subhead{Notes}
This system uses the line
\nextline
break mechanism
\nextline
\par
as well as
\begin{verbatim}
   a     verbatim
   table example
\end{verbatim}

and a line ending in colon:
\par
\begin{quotation}\noindent
This is in quotation format
\end{quotation}
\par
\par
\subhead{Notes on Fnodes}
This is a sample that includes
\begin{verbatim}
    verbatim 
    processing
\end{verbatim}

\par
\subhead{Keywords}
break, pointers
\nextline
\location{allfmt.c}
\endmanpage
\startmanpage
\mantitle{argtest1}{tex}{2/10/2000}
\manname{argtest1}
--- some examples of different argument list processing 
\subhead{Synopsis}
\startvb\begin{verbatim}
#include "tools.h"
#include "system/system.h"
void argtest1(int arg1, int arg2, int arg3, int arga, int argz)
\end{verbatim}
\endvb

\subhead{List1}
\startarg{arg1 }{first
}
\startarg{arg2 }{second
}
\startarg{arg3 }{third
}
\par
\subhead{List2}
\startarg{arga }{initial
}
\par
\subhead{List3}
\startarg{argz }{last
}
\par
\location{tst3.c}
\endmanpage
\startmanpage
\mantitle{Routine}{tex}{2/10/2000}
\manname{Routine}
--- foo is a routine  
\subhead{Synopsis}
\startvb\begin{verbatim}
#include "tools.h"
#include "system/system.h"
Routine( bar, another )
double *bar;
int another;
\end{verbatim}
\endvb

\subhead{Input Parameter}
\startarg{bar }{drinks
}
\startarg{another }{foo
}
\par
\location{test.nam}
\endmanpage
\startmanpage
\mantitle{Name}{tex}{2/10/2000}
\manname{Name}
--- is a macro for something 
\subhead{Synopsis}
\startvb\begin{verbatim}
Name(bar,foo)
int bar;
char foo;
\end{verbatim}
\endvb

\location{test.nam}
\endmanpage
\startmanpage
\mantitle{Name2}{tex}{2/10/2000}
\manname{Name2}
--- is a macro for something else 
\subhead{Synopsis}
\startvb\begin{verbatim}
Name2(bar,foo)
int bar;
char foo;
\end{verbatim}
\endvb
More stuff
\location{test.nam}
\endmanpage
\startmanpage
\mantitle{foo1}{tex}{2/10/2000}
\manname{foo1}
--- test  
\subhead{Synopsis}
\startvb\begin{verbatim}
#include "tools.h"
#include "system/system.h"

void foo1( a, b, c )
int a;
char b;
void (*c)();
\end{verbatim}
\endvb

\location{tstpgm.c}
\endmanpage
\startmanpage
\mantitle{foo2}{tex}{2/10/2000}
\manname{foo2}
--- test  
\subhead{Synopsis}
\startvb\begin{verbatim}
#include "tools.h"
#include "system/system.h"

struct tm *foo2( a, b )
MyType *a;
short b;
\end{verbatim}
\endvb

\location{tstpgm.c}
\endmanpage
\startmanpage
\mantitle{foo3}{tex}{2/10/2000}
\manname{foo3}
--- test 
\subhead{Synopsis}
\startvb\begin{verbatim}
#include "tools.h"
#include "system/system.h"

int foo3()
\end{verbatim}
\endvb

\location{tstpgm.c}
\endmanpage
\startmanpage
\mantitle{SYGetArchType}{tex}{2/10/2000}
\manname{SYGetArchType}
--- Return a standardized architecture type for the machine that is executing this routine.  This uses uname where possible, but may modify the name (for example, sun4 is returned for all sun4 types). 
\subhead{Synopsis}
\startvb\begin{verbatim}
#include "tools.h"
#include "system/system.h"
void SYGetArchType( str, slen )
char *str;
int  slen;
\end{verbatim}
\endvb

\subhead{Input Parameter}
slen - length of string buffer
\subhead{Output Parameter}
\startarg{str }{string area to contain architecture name.  Should be at least 
10 characters long.
}
\par
\subhead{Notes}
This is a long block of text that contains a number of lines
and has several sentances.  Such as this one, which is the
second of many.  Well, maybe not many.  Maybe four or
five sentances?
\par
\location{tstpgm.c}
\endmanpage
\startmanpage
\mantitle{foobar}{tex}{2/10/2000}
\manname{foobar}
--- test program C only 
\subhead{Synopsis}
\startvb\begin{verbatim}
#include "tools.h"
#include "system/system.h"
int foobar( a )
void **a;
\end{verbatim}
\endvb
This is used as an example of C - only (should be ignored)
\location{tstpgm.c}
\endmanpage
\startmanpage
\mantitle{simple}{tex}{2/10/2000}
\manname{simple}
--- test 
\subhead{Synopsis}
\startvb\begin{verbatim}
#include "tools.h"
#include "system/system.h"

int simple(int a, double b)
\end{verbatim}
\endvb

\location{tstpgma.c}
\endmanpage
\startmanpage
\mantitle{foo1}{tex}{2/10/2000}
\manname{foo1}
--- test  
\subhead{Synopsis}
\startvb\begin{verbatim}
#include "tools.h"
#include "system/system.h"

void foo1( int a, char b, void (*c)() )
\end{verbatim}
\endvb

\location{tstpgma.c}
\endmanpage
\startmanpage
\mantitle{foo1a}{tex}{2/10/2000}
\manname{foo1a}
--- test  
\subhead{Synopsis}
\startvb\begin{verbatim}
#include "tools.h"
#include "system/system.h"

void foo1a( int a, char b, void (*c)(int, char, MyType *a) )
\end{verbatim}
\endvb

\location{tstpgma.c}
\endmanpage
\startmanpage
\mantitle{foo2}{tex}{2/10/2000}
\manname{foo2}
--- test  
\subhead{Synopsis}
\startvb\begin{verbatim}
#include "tools.h"
#include "system/system.h"

struct tm *foo2( MyType *a, short b )
\end{verbatim}
\endvb

\location{tstpgma.c}
\endmanpage
\startmanpage
\mantitle{foo3}{tex}{2/10/2000}
\manname{foo3}
--- test 
\subhead{Synopsis}
\startvb\begin{verbatim}
#include "tools.h"
#include "system/system.h"

int foo3()
\end{verbatim}
\endvb

\location{tstpgma.c}
\endmanpage
\startmanpage
\mantitle{foo3a}{tex}{2/10/2000}
\manname{foo3a}
--- test ANSI should use this for empty arg 
\subhead{Synopsis}
\startvb\begin{verbatim}
#include "tools.h"
#include "system/system.h"

int foo3a(void)
\end{verbatim}
\endvb

\location{tstpgma.c}
\endmanpage
\startmanpage
\mantitle{Tst2{\tt \char`\_}start}{tex}{2/10/2000}
\manname{Tst2{\tt \char`\_}start}
--- Sample C only routine 
\subhead{Synopsis}
\startvb\begin{verbatim}
#include "tools.h"
#include "system/system.h"
void Tst2_start( a, b )
int a, b;
\end{verbatim}
\endvb
This example uses
\linebreak   The \$ symbol to do verbatim output\linebreak   up to here
\par
\centerline{\epsf{foo.ps}}\begin{center}\bf
\par
Caption for figure
\end{center}
\par
\par
\par
\location{tstpgmc.c}
\endmanpage
\startmanpage
\mantitle{foo1}{tex}{2/10/2000}
\manname{foo1}
--- test  
\subhead{Synopsis}
\startvb\begin{verbatim}
void foo1( a, b, c )
int a;
char b;
void (*c)();
\end{verbatim}
\endvb

\location{tstpgm.h}
\endmanpage
\startmanpage
\mantitle{foo2}{tex}{2/10/2000}
\manname{foo2}
--- test  
\subhead{Synopsis}
\startvb\begin{verbatim}
struct tm *foo2( a, b )
MyType *a;
short b;
\end{verbatim}
\endvb

\location{tstpgm.h}
\endmanpage
\startmanpage
\mantitle{foo3}{tex}{2/10/2000}
\manname{foo3}
--- test 
\subhead{Synopsis}
\startvb\begin{verbatim}
int foo3()
\end{verbatim}
\endvb

\subhead{Notes}
foo
\location{tstpgm.h}
\endmanpage
